\documentclass[11pt,nocut]{article}

\usepackage{../latex_style/packages}
\usepackage{../latex_style/notations}
%\externaldocument{../lecture_02/lecture_02}
\externaldocument{../lecture_07/lecture_07}


\title{\vspace{-2.0cm}%
	Optimization and Computational Linear Algebra for Data Science\\
Lecture 12: Gradient descent}
\author{Léo \textsc{Miolane} \ $\cdot$ \ \texttt{leo.miolane@gmail.com}}
\date{\today}

\begin{document}
\maketitle
\textbf{Warning:}
\emph{This material is not meant to be lecture notes. It only gathers the main concepts and results from the lecture, without any additional explanation, motivation, examples, figures...
}


\section{Gradient descent}

We aim at minimizing a differentiable function $f: \R^n \to \R$.
Given an initial point $x^{(0)} \in \R^n$, the gradient descent algorithm follows the updates:
\begin{equation}\label{eq:gradient_step}
x^{(k+1)} = x^{(k)} - \alpha_k \nabla f(x^{(k)}),
\end{equation}
where the step-size $\alpha_k$ remains to be determined.
The step \eqref{eq:gradient_step} is a very natural strategy to minimize $f$, since $-\nabla f(x)$ is the direction of steepest descent at $x$. Since $f(x+h) = f(x) + \langle \nabla f(x), h \rangle + o(\|h\|)$ we have
\begin{align*}
f(x^{(k+1)}) 
&= f(x^{(k)}) - \alpha_k \| \nabla f(x^{(k)}) \|^2 + o(\alpha_k) \\
&< f(x^{(k)}) 
\end{align*}
for $\alpha_k$ small enough (provided that $\nabla f(x^{(k)}) \neq 0$).
Hence is the step-sizes $\alpha_k$ are chosen very small, the sequence $(f(x^{(k)}))_{k \geq 0}$ is decreasing!


\begin{algorithm}
\caption{Gradient descent}
\begin{algorithmic}[1]
	\Statex \textbf{Input:} Graph Laplacian $L$, number of clusters $k$
	\State Compute the first $k$ eigenvectors $v_1, \dots, v_k$ of the Laplacian matrix $L$.
	\State Associate to each node $i$ the vector $x_i = (v_2(i), \dots, v_k(i)) \in \R^{k-1}$.
	\State Cluster the points $x_1, \dots, x_n$ with (for instance) the $k$-means algorithm.
\end{algorithmic}
\end{algorithm}

\section{Newton's method}


\section*{Further reading}


\vspace{1cm}
\centerline{\pgfornament[width=7cm]{71}}


\bibliographystyle{plain}
\bibliography{../references.bib}
\end{document}
