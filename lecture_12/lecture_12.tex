\documentclass[11pt,nocut]{article}

\usepackage{../latex_style/packages}
\usepackage{../latex_style/notations}
%\externaldocument{../lecture_02/lecture_02}
\externaldocument{../lecture_07/lecture_07}


\title{\vspace{-2.0cm}%
	Optimization and Computational Linear Algebra for Data Science\\
Lecture 12: Gradient descent}
\author{Léo \textsc{Miolane} \ $\cdot$ \ \texttt{leo.miolane@gmail.com}}
\date{\today}

\begin{document}
\maketitle
\textbf{Warning:}
\emph{This material is not meant to be lecture notes. It only gathers the main concepts and results from the lecture, without any additional explanation, motivation, examples, figures...
}


\begin{center}
In these notes, $f$ denotes a twice differentiable \textbf{convex} function from $\R^n$ to $\R$.
\end{center}

\section{Gradient descent}

Given an initial point $x_0 \in \R^n$, the gradient descent algorithm follows the updates:
\begin{equation}\label{eq:gradient_step}
x_{t+1} = x_t - \alpha_t \nabla f(x_t),
\end{equation}
where the step-size $\alpha_t$ remains to be determined.
The step \eqref{eq:gradient_step} is a very natural strategy to minimize $f$, since $-\nabla f(x)$ is the direction of steepest descent at $x$. Since $f(x+h) = f(x) + \langle \nabla f(x), h \rangle + o(\|h\|)$ we have
\begin{align*}
f(x_{t+1}) 
&= f(x_t) - \alpha_t \| \nabla f(x_t) \|^2 + o(\alpha_t) \\
&< f(x_t) 
\end{align*}
for $\alpha_t$ small enough (provided that $\nabla f(x_t) \neq 0$).
Hence is the step-sizes $\alpha_t$ are chosen very small, the sequence $(f(x_t))_{k \geq 0}$ is decreasing!
However, if $\alpha_t$ are too small, the algorithm may never converge.

\subsection{Convergence analysis}

\textbf{Notation}: Given a symmetric matrix $M$ we will denote by $\lambda_{\rm min}(M)$ and $\lambda_{\rm max}(M)$ the smallest and largest eigenvalues of $M$. 
\begin{definition}
	For $L,\mu >0$, we say that a twice-differentiable convex function $f: \R^n \to \R$ is
	\begin{itemize}
		\item $L$-smooth if for all $x \in \R^n$, $\lambda_{\rm max}(H_f(x)) \leq L$.
		\item $\mu$-strongly convex if for all $x \in \R^n$, $\lambda_{\rm min}(H_f(x)) \geq \mu$.
	\end{itemize}
\end{definition}


\begin{theorem}\label{th:gradient1}
	Assume that $f$ is $L$-smooth and that $f$ admits a (global) minimizer $x^* \in \R^n$. Then
	the gradient descent iterates \eqref{eq:gradient_step} with constant step-size $\alpha_k = 1/L$ verify
	$$
	f(x_t) - f(x^*) \leq \frac{2 L \| x_0 - x^* \|^2}{t+4}.
	$$
\end{theorem}
See Section 2.1.5 from \cite{nesterov2018lectures} for a proof.

\paragraph{Why did we used step sizes of $1/L$ ?} If $f$ is $L$-smooth one can prove (see Homework~9) that for all $x,h \in \R^n$:
\begin{equation}\label{eq:upperL}
f(x+h) \leq f(x) + \langle \nabla f(x) , h \rangle + \frac{L}{2} \|h\|^2.
\end{equation}
Then, one can check (exercise!) that when $x$ is fixed, the minimum of the right-hand side is minimum for $h = - \frac{1}{L} \nabla f(x)$.

\begin{theorem}\label{th:gradient2}
	Assume that $f$ is $L$-smooth and $\mu$-strongly convex. 
	Then $f$ admits a unique minimizer global $x^*$ and
	the gradient descent iterates \eqref{eq:gradient_step} with constant step-size $\alpha_k = 1/L$ verify
	$$
	f(x_t) - f(x^*) \leq \Big(1-\frac{\mu}{L}\Big)^t (f(x_0) - f(x^*)).
	$$
\end{theorem}
\begin{remark}
	The ratio $\kappa = \frac{L}{\mu} \in (0,1]$ is called the condition number. The smaller the condition number, the faster the convergence.
\end{remark}
\begin{remark}
	The $\mu$-strong convexity of $f$ implies that for all $x \in \R^n$,
	$$
	\frac{\mu}{2} \|x - x^* \|^2 \leq f(x)-f(x^*).
	$$
	Combining this with Theorem \ref{th:gradient2} gives a bound of the distance to the minimizer $x^*$:
	$$
	\|x_t - x^* \|^2 \leq \frac{2}{\mu}\Big(1-\frac{\mu}{L}\Big)^t (f(x_0) - f(x^*)).
	$$
\end{remark}

\begin{proof}
	Let $t \geq 0$. Applying \eqref{eq:upperL} for $x=x_t$ and $h=x_t - L^{-1} \nabla f(x_t)$, we get
	$$
	f(x_{t+1}) 
	\leq f(x_t) - \frac{1}{L} \|\nabla f(x_t) \|^2 + \frac{1}{2L}\|\nabla f(x_t) \|^2
	= f(x_t) - \frac{1}{2L} \|\nabla f(x_t) \|^2.
	$$
Now, since $f$ is $\mu$-strongly convex, we have for all $x \in \R^n$
$$
f(x) - f(x^*) \leq 2 \mu \|\nabla f(x)\|^2.
$$
We get that $f(x_{t+1}) \leq f(x_t) - \frac{\mu}{L}(f(x_t) - f(x^*))$, hence
$$
f(x_{t+1}) - f(x^*) \leq (1 - \frac{\mu}{L})(f(x_t) - f(x^*)),
$$
from which the theorem follows.
\end{proof}

\subsection{Choosing the step size in practice}

In practice, one may not have access to $L$ and need hence to choose the step size $\alpha_t$. A popular method is the so-called ``backtracking line search'' a goes as follows.
Fix a parameter $\beta \in (0,1)$.
Start with $\alpha = 1$ and while
$$
f\big(x_t - \alpha \nabla f (x_t) \big) > f(x_t) - \frac{\alpha}{2} \|\nabla f(x_t) \|^2,
$$
update $\alpha =\beta \alpha$. Then choose $\alpha_t = \alpha$.

\subsection{Accelerated gradient method}

\paragraph{Gradient descent with momentum.}

Also known as ``heavy ball'' method, this scheme was introduced by Polyak in 1964.
This is a way to prevent zigzagging trajectories when doing gradient descent by adding a momentum term:
\begin{align*}
	x_{t+1} = x_t + v_t
	\quad \text{where} \quad
	v_t = \alpha_t v_{t-1} - \beta_t \nabla f(x_{t-1}),
\end{align*}
for some $\alpha_t,\beta_t$. The idea is that the 

\paragraph{Nesterov's accelerated gradient descent.}

Nesterov's accelerated gradient descent is an amelioration of idea of momentum.
\begin{align*}
	x_{t+1} = x_t + v_t
	\quad \text{where} \quad
	v_t = \alpha_t v_{t-1} - \beta_t \nabla f(x_{t-1} + \alpha_t v_t)
\end{align*}

When $\alpha_t,\beta_t$ are properly chosen, it improves on the convergence rates of gradient descent (given by Theorems~\ref{th:gradient1}-\ref{th:gradient2}). Namely:
\begin{itemize}
	\item if $f$ is $L$-smooth and if its minimum is attained at some $x^*$, then for $\alpha_t = \frac{t-1}{t+2}$ and $\beta_t = 1/L$ we have
		$$
		f(x_t) - f(x^*) \leq \frac{2L \|x_0-x^*\|^2}{(t+1)^2}.
		$$
	\item if $f$ is $L$-smooth and $\mu$-strongly convex, then for $\alpha_t = \frac{1-\sqrt{\mu/L}}{1+\sqrt{\mu/L}}$ and $\beta_t = 1/L$ we have
		$$
		f(x_t) - f(x^*) \leq L \|x_0-x^*\|^2 \Big(1-\sqrt{\mu/L}\Big)^t.
		$$
\end{itemize}
See for instance \cite{schmidt2011convergence} for proofs of these results.


\section{Newton's method}

\subsection{Newton's method}
We assume here that $f$ is $\mu$-strongly convex and $L$-smooth.
Newton's method performs updates according to
\begin{equation}\label{eq:newton}
	x_{t+1} = x_t - H_f(x_t)^{-1} \nabla f(x_t).
\end{equation}
The (important!) difference with gradient descent is that the step-size $\alpha_k$ is now replaced by the inverse\footnote{The Hessian of $f$ if indeed invertible at all $x$ since its smallest eigenvalue is always greater than $\mu >0$.} of the Hessian of $f$. The idea begin Newton's method is to minimize the second order approximation of $f$ at $x_t$ :
\begin{equation}\label{eq:sec}
	f(x_t+h) \simeq f(x_t) + \langle \nabla f(x_t), h \rangle + \frac{1}{2} h^{\sT} H_f(x_t) h
\end{equation}
with respect to $h$ and then choose $x_{t+1} = x_t + h$. It is an easy exercise to see that the minimizer of the right-hand side of \eqref{eq:sec} is $h=- H_f(x_t)^{-1} \nabla f(x_t)$, leading to the recursion \eqref{eq:newton}.
\\

It can be shown (see for instance \cite{boyd2004convex}) that for $t$ large enough
\begin{equation}\label{eq:newton}
\|x_t - x^* \|^2 \leq C e^{-\rho 2^t},
\end{equation}
where $C,\rho$ are constants depending on $f$ and $x_0$. We say that Newton's method converges \emph{quadratically} to the minimizer $x^*$. Newton's method is much faster than gradient descent, whose speed (given by Theorem \ref{th:gradient2}) is of order $C' e^{-\sqrt{\mu/L} \, t}$.


\subsection{Quasi-Newton methods}

The main drawback of Newton's method is its computational complexity. Each step of the method require to compute the inverse of the $n \times n$ Hessian matrix of $f$ at $x_t$, which require $O(n^3)$ operations. This makes Newton's method unpractical for large scale applications.
\\

Quasi-Newton methods have been developed to face these limitations. 
The idea behind quasi-Newton methods is to try to mimic the inverse Hessian $H_f(x_t)^{-1}$ by a sequence of symmetric positive semidefinite matrices $(Q_t)_{t \geq 0}$ that are recursively computed in an efficient way. We refer to Chapter 6 of \cite{nocedal2006numerical} for a detailed introduction to this topic.
%\\

%A starting point is to say that when $x_t$ and $x_{t-1}$ are close,
%$$
%\nabla f(x_t) \simeq \nabla f(x_{t-1}) + H_f(x_t)(x_t - x_{t-1}).
%$$
%Hence, writing $\Delta x = x_t - x_{t-1}$ and $\Delta g = \nabla f(x_t) - \nabla f(x_{t-1})$, we have $\Delta g \simeq H_f(x_t)\Delta x$. Hence it is natural to impose that $Q_{t}$ verifies $Q_{t} \Delta x = \Delta g$.
%The BFGS method choses


\section*{Further reading}

See chapter 9 of \cite{boyd2004convex} for more background on gradient descent and Newton's method.
%See chapter 6 of \cite{nocedal2006numerical} for a detailed introduction to quasi-Newton methods.

\vspace{1cm}
\centerline{\pgfornament[width=7cm]{71}}


\bibliographystyle{plain}
\bibliography{../references.bib}
\end{document}
