\documentclass{beamer}

\usepackage{../../latex_style/beamerthemeExecushares}
\usepackage{../../latex_style/notations}

\title{Lecture 4.1: Norms}
\subtitle{Optimization and Computational Linear Algebra for Data Science}
\author{Léo Miolane}
\date{}

\setcounter{showSlideNumbers}{1}

\begin{document}
\setcounter{showProgressBar}{0}
\setcounter{showSlideNumbers}{0}

\frame{\titlepage}

\setcounter{framenumber}{0}
\setcounter{showSlideNumbers}{1}
\begin{frame}[t]{Introduction: the Euclidean norm}
	\begin{definition}
		We define the Euclidean norm of $x=(x_1, \dots, x_n) \in \R^n$ as:
		$$
		\| x\|_2 = \sqrt{x_1^2 + \cdots + x_n^2}.
		$$
	\end{definition}


\end{frame}

\begin{frame}[t]{General norms}
	Let $V$ be a vector space.
\begin{definition}
	A norm $\| \cdot \|$ on $V$ is a function from $V$ to $\R_{\geq 0}$ that verifies the following points:
	\begin{enumerate}
		\item \emph{Homogeneity}: $\| \alpha v \| = |\alpha|\times \| v\|$ for all $\alpha \in \R$ and  $v \in V$.
		\item \emph{Triangular inequality}: $\|u + v\| \leq \|u\| + \|v\|$ for all $u,v \in V$.
		\item \emph{Positive definiteness}: if $\|v\| = 0$ for some $v \in V$, then $v=0$.
	\end{enumerate}
\end{definition}
\end{frame}


\begin{frame}[t]{Other examples}
	\begin{itemize}
\item The $\ell_1$ norm 
	\end{itemize}
		$$\|x\|_1 \ \defeq \ \sum_{i=1}^n |x_i| \ = \  |x_1| + \cdots + |x_n|.
\end{frame}
\begin{frame}[t]{Other examples}
	\begin{itemize}
\item The $\ell_p$ norm, for $p \geq 1$
	$$\|x\|_p \ \defeq \ \Big( |x_1|^p + \cdots + |x_n|^p \Big)^{1/p}.$$
		\item The infinity-norm 
			$$\|x \|_{\infty} \defeq \max(|x_1|, \dots, |x_n|).$$
	\end{itemize}
\end{frame}
\begin{frame}[t]{Balls}
	For each of the norms $\| \cdot \|_2$, $\| \cdot \|_1$ and $\| \cdot \|_{\infty}$, draw the <<ball>>:
	$$
	B = \big\{ x \in \R^2 \, \big| \, \|x\| \leq 1 \big\}.
	$$
\end{frame}

\end{document}
