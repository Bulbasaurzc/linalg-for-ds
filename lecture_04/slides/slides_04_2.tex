\documentclass{beamer}

\usepackage{../../latex_style/beamerthemeExecushares}
\usepackage{../../latex_style/notations}

\title{Lecture 4.2: Inner product}
\subtitle{Optimization and Computational Linear Algebra for Data Science}
\author{Léo Miolane}
\date{}

\setcounter{showSlideNumbers}{1}

\begin{document}
\setcounter{showProgressBar}{0}
\setcounter{showSlideNumbers}{0}

\frame{\titlepage}

\setcounter{framenumber}{0}
\setcounter{showSlideNumbers}{1}
\begin{frame}[t]{The Euclidean dot product}
	\vspace{-0.3cm}
	\begin{definition}
		We define the Euclidean dot product of two vectors $x$ and $y$ of $\R^n$ as:
		\vspace{-0.3cm}
		$$
		x \cdot y  \ = \ \sum_{i=1}^n x_i \, y_i \ = \ x_1 y_1 + \cdots + x_n y_n.
		\vspace{-0.3cm}
		$$
	\end{definition}
\end{frame}

\begin{frame}[t]{Inner product}
	Let $V$ be a vector space.
	\begin{definition}
		An inner product on $V$ is a function $\langle \cdot, \cdot \rangle$ from $V \times V$ to $\R$ that verifies the following points:
		\begin{enumerate}
			\item \emph{Symmetry}: $\langle u, v \rangle = \langle v, u\rangle$ for all $u,v \in V$.
			\item \emph{Linearity}: $\langle u+v, w \rangle = \langle u, w\rangle + \langle v, w\rangle$ and $\langle \alpha v, w \rangle = \alpha \langle v, w \rangle$ for all $u,v,w \in V$ and $\alpha \in \R$.
			\item \emph{Positive definiteness}: $\langle v, v\rangle \geq 0$ with equality if and only if $v = 0$.
		\end{enumerate}
	\end{definition}
\end{frame}


\begin{frame}[t]{Other example}
	If $V$ is the set of all random variables (on a probability space $\Omega$) that have a finite second moment, then 
	$$\langle X, Y \rangle \ \defeq  \ \E[XY]$$
	is an inner product on $V$.
\end{frame}

\begin{frame}[t]{Norm induced by an inner product}
	\vspace{-0.3cm}
	\begin{block}{Proposition}
		If $\langle \cdot, \cdot \rangle$ is an inner product on $V$ then 
		$$
		\| v \| \defeq \sqrt{\langle v, v \rangle}
		$$ 
		is a norm on $V$. We say  that the norm $\| \cdot \|$ is induced by the inner product $\langle \cdot, \cdot \rangle$.
	\end{block}
\end{frame}

\begin{frame}[t]{Example}
	Consider again the set $V$ of all random variables (on a probability space $\Omega$) that have a finite second moment, with the inner product:
	$$\langle X, Y \rangle \ \defeq  \ \E[XY].$$
\end{frame}

\begin{frame}[t]{Cauchy Schwarz inequality}
\begin{theorem}[Cauchy-Schwarz inequality]
	Let $\| \cdot \|$ be the norm induced by the inner product $\langle \cdot , \cdot \rangle$ on the vector space $V$. Then for all $x,y \in V$:
	\begin{equation}\label{eq:cauchy_schwarz}
	| \langle x,y \rangle | \leq \|x\| \, \|y\|.
	\end{equation}
	Moreover, there is equality in \eqref{eq:cauchy_schwarz} if and only if $x$ and $y$ are linearly dependent, i.e.\ $x = \alpha y$ or $y = \alpha x$ for some $\alpha \in \R$.
\end{theorem}
\end{frame}


\begin{frame}[t]{Examples}
\end{frame}
\end{document}
