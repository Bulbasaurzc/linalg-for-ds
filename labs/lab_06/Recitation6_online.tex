\documentclass{beamer}

\usepackage{../../latex_style/beamerthemeExecushares}
\usepackage{../../latex_style/notations}

\title{Recitation 6}
\author{Carles Domingo}
\date{Fall 2020}


\begin{document}

\frame{\titlepage} 

\setcounter{showProgressBar}{0}
\setcounter{showSlideNumbers}{1}

\begin{frame}[t]
\frametitle{Markov Chains}
\begin{definition}[Markov chain]
	A sequence of random variables $(X_0, X_1, \dots)$ is a Markov chain with state space $E$ and ``transition matrix'' $P$ if for all $t \geq 0$, 
	$$
	\P\big( X_{t+1} = y \, \big| \, X_0= x_0, \dots, X_t = x_t ) = P(x_t,y)
	$$
	for all $x_0, \dots, x_t$ such that $\P(X_0 = x_0, \dots, X_t = x_t) >0$.
\end{definition}
Stochastic matrix: $P_{ij} \geq 0$, $\sum_{i=1}^n P_{ij} = 1$ for all $1 \leq j \leq n$.
\begin{definition}[Invariant measure]
	A vector $\mu \in \Delta_n$ is called an invariant measure for the transition matrix $P$ if $\mu = P \mu$, i.e.\ if $\mu$ is an eigenvector of $P$ associated with the eigenvalue $1$.
\end{definition}
\end{frame}

\begin{frame}[t]
\frametitle{Perron-Frobenius theorem}
\begin{theorem}[Perron-Frobenius, stochastic case]\label{th:perron_frobenius}
	Let $P$ be a stochastic matrix such that there exists $k \geq 1$ such that all the entries of $P^k$ are strictly positive. Then the following holds:
	\begin{enumerate}
		\item $1$ is an eigenvalue of $P$ and there exists an eigenvector $\mu \in \Delta_n$ associated to $1$.
		\item The eigenvectors associated to $1$ are unique up to scalar multiple (i.e.\ $\Ker(P-\Id) = \Span(\mu)$).
		\item For all $x \in \Delta_n$, $P^t x \xrightarrow[t \to \infty]{} \mu$.
	\end{enumerate}
\end{theorem}
Is the condition "there exists $k \geq 1$ such that all the entries of $P^k$ are strictly positive" necessary? Let's see!
\end{frame}

\iffalse
\begin{frame}[t]
\frametitle{Questions: Counterexamples}
\begin{definition}[Irreducible Markov chain]
If for all $1 \leq i,j \leq n$, there exists $k \geq 1$ such that $P^{k}_{ij} > 0$, we say that the Markov chain is irreducible.
\end{definition}
\begin{definition}[Aperiodic Markov chain]
If for all $1 \leq i \leq n$, we have $\text{gcd}(\{k | P_{ii}^{k} > 0\}) = 1$, we say that the Markov chain is aperiodic.
\end{definition}
\begin{enumerate}
\item Show that if "there exists $k \geq 1$ such that all the entries of $P^k$ are strictly positive", then the Markov chain is irreducible and aperiodic. The converse is also true but harder to prove (come to office hours if you want to know!).
\item Show that irreducible non-aperiodic Markov chains have no invariant measure.
\item Show that non-irreducible aperiodic Markov chains have several invariant measures.
\end{enumerate}
\end{frame}

\begin{frame}[t]
\frametitle{Questions: Counterexamples}
1. Show that if "there exists $k \geq 1$ such that all the entries of $P^k$ are strictly positive", then the Markov chain is irreducible and aperiodic. The converse is also true but harder to prove (come to office hours if you want to know!).
\pause
\end{frame}

\begin{frame}[t]
\frametitle{Questions: Counterexamples}
2. Show that irreducible non-aperiodic Markov chains have no invariant measure.
\pause
\end{frame}

\begin{frame}[t]
\frametitle{Questions: Counterexamples}
3. Show an example of a non-irreducible aperiodic Markov chains that has several invariant measures.
\pause
\end{frame}
\fi

\begin{frame}[t]
\frametitle{Questions: Counterexamples}
\vspace{-10pt}
\begin{definition}[Irreducible Markov chain]
If for all $1 \leq i,j \leq n$, there exists $k \geq 1$ such that $P^{k}_{ij} > 0$, we say that the Markov chain is irreducible.
\end{definition}
\begin{enumerate}
\item Show that the assumption "there exists $k \geq 1$ such that all the entries of $P^k$ are strictly positive" implies that the Markov chain is irreducible.
\item Find an example of a non-irreducible Markov chain for which several invariant measures exist.
\item But irreducibility is not enough for the Perron-Frobenius statements to hold. Show that a Markov chain with transition matrix
\begin{align*}
P = 
\begin{bmatrix}
0 & 1 \\
1 & 0
\end{bmatrix}
\end{align*}
is irreducible but does not fulfill "for all $x \in \Delta_2$,  $P^t x \xrightarrow[t \to \infty]{} \mu$".
\end{enumerate}
\end{frame}

\begin{frame}[t]
\frametitle{Questions: Counterexamples}
\vspace{-10pt}
\begin{definition}[Irreducible Markov chain]
If for all $1 \leq i,j \leq n$, there exists $k \geq 1$ such that $P^{k}_{ij} > 0$, we say that the Markov chain is irreducible.
\end{definition}
1. Show that the assumption "there exists $k \geq 1$ such that all the entries of $P^k$ are strictly positive" implies that the Markov chain is irreducible.
\end{frame}

\begin{frame}[t]
\frametitle{Questions: Counterexamples}
\vspace{-10pt}
\begin{definition}[Irreducible Markov chain]
If for all $1 \leq i,j \leq n$, there exists $k \geq 1$ such that $P^{k}_{ij} > 0$, we say that the Markov chain is irreducible.
\end{definition}
2. Find an example of a non-irreducible Markov chain for which several invariant measures exist.
\pause
\end{frame}

\begin{frame}[t]
\frametitle{Questions: Counterexamples}
\vspace{-10pt}
\begin{definition}[Irreducible Markov chain]
If for all $1 \leq i,j \leq n$, there exists $k \geq 1$ such that $P^{k}_{ij} > 0$, we say that the Markov chain is irreducible.
\end{definition}
3. But irreducibility is not enough for the Perron-Frobenius statements to hold. Show that a Markov chain with transition matrix
\begin{align*}
P = 
\begin{bmatrix}
0 & 1 \\
1 & 0
\end{bmatrix}
\end{align*}
is irreducible but does not fulfill "for all $x \in \Delta_2$,  $P^t x \xrightarrow[t \to \infty]{} \mu$".
\pause
\end{frame}

\begin{frame}[t]
\frametitle{Questions: Counterexamples}
4. Remember from the lecture that the PageRank algorithm actually computes the invariant measure of the transition matrix
\begin{align*}
G = \alpha P + \frac{1-\alpha}{N} 
\begin{bmatrix} 
1 & \cdots & 1 \\
\vdots & \cdots & \vdots \\
1 & \cdots & 1
\end{bmatrix}
\end{align*}
with $\alpha \approx 0.85$. Given the previous questions, what would be the problems in taking $\alpha = 1$? 
\pause
\end{frame}

\begin{frame}[t]
\frametitle{Questions: Stochastic matrices}
Remember that  $P$ is a stochastic matrix when $P_{i,j} \geq 0$ for all $1 \leq i,j \leq n$ and $\sum_{i=1}^n P_{i,j} = 1$ for all $j$. 
\begin{enumerate}
\item Show that $1$ is an eigenvalue of $P$.
\item Show that all eigenvalues of $P$ have absolute value less or equal than $1$.
\end{enumerate}
\pause
\pause
\end{frame}

\begin{frame}[t]
\frametitle{Questions: Random walks}
Let us consider a variant of PageRank in which the edges are non-oriented, i.e. if page $i$ contains a link to page $j$, then page $j$ contains a link to page $i$. If we define the transition 
\begin{align*}
P_{i,j} =
\begin{cases}
1/\text{deg}(j) &\text{if link } i \leftrightarrow j  \\
0 &\text{otherwise}
\end{cases}
\end{align*}
\begin{enumerate}
\item Show that $\pi$ defined as $\pi_j = \text{deg}(j)$ is an eigenvector of $P$ of eigenvalue 1.
\item Conclude that $x \in \Delta_n$,  $P^t x \xrightarrow[t \to \infty]{} \tilde{\pi}$ if the Perron-Frobenius assumption holds, where $\tilde{\pi}_i = \pi_i / (\sum_{j=1}^n \pi_j)$ is the scaled multiple of $\pi$ belonging to $\Delta_n$.
\item Extra Question: Show that if each page has a link to itself and for any pair of pages $i,j$, there is a path of linked pages joining $i$ and $j$, the Perron-Frobenius assumption holds.
\end{enumerate}
\end{frame}

\begin{frame}[t]
\frametitle{Questions: Random walks}
\vspace{-10pt}
\begin{align*}
P_{i,j} =
\begin{cases}
1/\text{deg}(j) &\text{if link } i \leftrightarrow j  \\
0 &\text{otherwise}
\end{cases}
\end{align*}
1. Show that $\pi$ defined as $\pi_j = \text{deg}(j)$ is an eigenvector of $P$ of eigenvalue 1.
\end{frame}

\begin{frame}[t]
\frametitle{Questions: Random walks}
\vspace{-10pt}
\begin{align*}
P_{i,j} =
\begin{cases}
1/\text{deg}(j) &\text{if link } i \leftrightarrow j  \\
0 &\text{otherwise}
\end{cases}
\end{align*}
2. Conclude that $x \in \Delta_n$,  $P^t x \xrightarrow[t \to \infty]{} \tilde{\pi}$ if the Perron-Frobenius assumption holds, where $\tilde{\pi}_i = \pi_i / (\sum_{j=1}^n \pi_j)$ is the scaled multiple of $\pi$ belonging to $\Delta_n$.
\end{frame}

\begin{frame}[t]
\frametitle{Spectral theorem}
\begin{theorem}[Spectral theorem]
Let $A \in \R^{n \times n}$ be a symmetric matrix. Then, $A$ has $n$ orthogonal eigenvectors $q_1, \dots, q_n$ and we can write $A = Q \Lambda Q^\top$, where $Q = \begin{bmatrix} q_1 & \cdots & q_n \end{bmatrix}$ and $\Lambda$ is diagonal.
\end{theorem}

Remember that a matrix $A$ is diagonalizable iff it has $n$ linearly independent eigenvectors (equivalently $A = V \Lambda V^{-1}$). Thus, the spectral theorem says that symmetric matrices are diagonalizable in an orthogonal basis.
\end{frame}

\begin{frame}[t]
\frametitle{Questions: Spectral theorem}
1. Let $A, B \in \R^{n \times n}$ be symmetric matrices. Show that $AB = BA$ iff $A$ and $B$ diagonalize in the same basis. 
\pause
\end{frame}

\iffalse
\begin{frame}[t]
\frametitle{Questions: Spectral theorem}
%\vspace{-10pt}
\begin{theorem}[Courant-Fischer principle]
The eigenvalues $\lambda_1 \geq \cdots \geq \lambda_n$ of a symmetric matrix $A$ are given by
\begin{align*}
\lambda_k = \max_{\substack{S \subset \R^n \\ \text{dim}(S) = k}} \min_{\substack{x \in S \\ x \neq 0}} \frac{x^\top A x}{x^\top x} = \min_{\substack{S' \subset \R^n \\ \text{dim}(S') = n-k+1}} \max_{\substack{x \in S' \\ x \neq 0}} \frac{x^\top A x}{x^\top x}
\end{align*}
\end{theorem}
We will show this theorem. Seems like a lot, but we'll go step by step! We will only show the first equality as the argument for the second one is analogous.
\end{frame}

\begin{frame}[t]
\frametitle{Questions: Spectral theorem}
\vspace{-10pt}
\begin{align*}
\lambda_k = \max_{\substack{S \subset \R^n \\ \text{dim}(S) = k}} \min_{\substack{x \in S \\ x \neq 0}} \frac{x^\top A x}{x^\top x} = \min_{\substack{S' \subset \R^n \\ \text{dim}(S') = n-k+1}} \max_{\substack{x \in S' \\ x \neq 0}} \frac{x^\top A x}{x^\top x}
\end{align*}
Let $v_1, \cdots, v_n$ be the orthogonal basis of eigenvectors (resp.).
\begin{enumerate}[(i)]
\item Show that for $S_k = \text{span}(v_1, \dots, v_k)$, $\min_{x \in S, x \neq 0} \frac{x^\top A x}{x^\top x} \geq \lambda_k$.
\item Show that for all $x \in S_k' = \text{span}(v_k, \dots, v_n)$ such that $x \neq 0$, we have  $\frac{x^\top A x}{x^\top x} \leq \lambda_k$.
\item Check that for any subspace $S$ with $\text{dim}(S) = k$, $S \cup S_k' \neq \{0\}$.
\item Conclude.
\end{enumerate}
\end{frame}

\begin{frame}[t]
\frametitle{Questions: Spectral theorem}
\vspace{-15pt}
\begin{align*}
\lambda_k = \max_{\substack{S \subset \R^n \\ \text{dim}(S) = k}} \min_{\substack{x \in S \\ x \neq 0}} \frac{x^\top A x}{x^\top x} = \min_{\substack{S' \subset \R^n \\ \text{dim}(S') = n-k+1}} \max_{\substack{x \in S' \\ x \neq 0}} \frac{x^\top A x}{x^\top x}
\end{align*}
Let $v_1, \cdots, v_n$ be the orthogonal basis of eigenvectors (resp.).

(i) Show that for $S_k = \text{span}(v_1, \dots, v_k)$, $\min_{x \in S, x \neq 0} \frac{x^\top A x}{x^\top x} \geq \lambda_k$.
\end{frame}

\begin{frame}[t]
\frametitle{Questions: Spectral theorem}
\vspace{-15pt}
\begin{align*}
\lambda_k = \max_{\substack{S \subset \R^n \\ \text{dim}(S) = k}} \min_{\substack{x \in S \\ x \neq 0}} \frac{x^\top A x}{x^\top x} = \min_{\substack{S' \subset \R^n \\ \text{dim}(S') = n-k+1}} \max_{\substack{x \in S' \\ x \neq 0}} \frac{x^\top A x}{x^\top x}
\end{align*}
Let $v_1, \cdots, v_n$ be the orthogonal basis of eigenvectors (resp.).

(ii) Show that for all $x \in S_k' = \text{span}(v_k, \dots, v_n)$ such that $x \neq 0$, we have  $\frac{x^\top A x}{x^\top x} \leq \lambda_k$.
\end{frame}

\begin{frame}[t]
\frametitle{Questions: Spectral theorem}
\vspace{-15pt}
\begin{align*}
\lambda_k = \max_{\substack{S \subset \R^n \\ \text{dim}(S) = k}} \min_{\substack{x \in S \\ x \neq 0}} \frac{x^\top A x}{x^\top x} = \min_{\substack{S' \subset \R^n \\ \text{dim}(S') = n-k+1}} \max_{\substack{x \in S' \\ x \neq 0}} \frac{x^\top A x}{x^\top x}
\end{align*}
Let $v_1, \cdots, v_n$ be the orthonormal basis of eigenvectors (resp.).

(ii) Check that for any subspace $S$ with $\text{dim}(S) = k$, $S \cup S_k' \neq \{0\}$.

(iv) Conclude.
\end{frame}
\fi

\iffalse
\begin{frame}[t]
\frametitle{Extra question: Spectral theorem}
For a symmetric matrix, show how you can use a modification of Gaussian elimination to find a decomposition $A = V \Lambda V^\top$ for some diagonal matrix $\Lambda$.
\end{frame}
\fi

\begin{frame}[t]

\end{frame}

\begin{frame}[t]

\end{frame}

\begin{frame}[t]

\end{frame}

\begin{frame}[t]

\end{frame}

\end{document}


