\documentclass{beamer}
\usepackage{amsthm}
\usepackage[utf8]{inputenc}
\usetheme{Madrid}
\usepackage{../latex_style/packages_old}
\usepackage{../latex_style/notations_old}
\usepackage{outlines}
\usepackage{enumitem}
\renewenvironment{itemize}
\usefonttheme{serif}
\def\labelenumi{\theenumi}
\usefonttheme{serif}
\usepackage{comment}

%\setbeamertemplate{itemize items}[default]
%\setbeamertemplate{enumerate items}[default]

% define smaller font command
\newcommand*{\horzbar}{\rule[.5ex]{2.5ex}{0.5pt}}
\newcommand\fonteight{\fontsize{8}{9.6}\selectfont}
\newcommand\fontten{\fontsize{10}{1.2}\selectfont}

%define enumerate with periods
\renewenvironment{enumerate}%
{\begin{list}{\arabic{enumi}.}%  \langle ------ dot here
      {\setlength{\leftmargin}{2.5em}%
       \setlength{\itemsep}{-\parsep}%
       \setlength{\topsep}{-\parskip}%%
       \usecounter{enumi}}%
 }{\end{list}}
%define itemize with arrows
\renewenvironment{itemize}%
{\begin{list}{$\blacktriangleright$}%  \langle ------ dot here
      {\setlength{\leftmargin}{2.5em}%
       \setlength{\itemsep}{-\parsep}%
       \setlength{\topsep}{-\parskip}%%
       \usecounter{enumi}}%
 }{\end{list}}

%Information to be included in the title page:
\title{Recitation 8}
\author{Alex Dong}
\institute{CDS, NYU}
\date{Fall 2020}


\makeatletter
\setbeamertemplate{navigation symbols}{}
\setbeamertemplate{footline}{
  \leavevmode%
  \hbox{%
  \begin{beamercolorbox}[wd=.4\paperwidth,ht=2.25ex,dp=1ex,center]{author in head/foot}%
    \usebeamerfont{author in head/foot}\insertshortauthor\expandafter\ifblank\expandafter{\beamer@shortinstitute}{}{~~(\insertshortinstitute)}
  \end{beamercolorbox}%
  \begin{beamercolorbox}[wd=.3\paperwidth,ht=2.25ex,dp=1ex,center]{title in head/foot}%
    \usebeamerfont{title in head/foot}\insertshorttitle
  \end{beamercolorbox}%
  \begin{beamercolorbox}[wd=.3\paperwidth,ht=2.25ex,dp=1ex,right]{date in head/foot}%
    \usebeamerfont{date in head/foot}\insertshortdate{}\hspace*{2em}
    \insertframenumber{} / \inserttotalframenumber\hspace*{2ex} 
  \end{beamercolorbox}}%
  \vskip0pt%
}
\setbeamertemplate{navigation symbols}{}
\makeatother

\begin{document}
%1
\frame{\titlepage} 
%2

\begin{frame}
\frametitle{Singular Value Decomposition}
\begin{itemize}
\item $A = U\Sigma V^T$
\begin{itemize}
\item $A, \Sigma \in \R^{n \times m},\quad U \in \R^{n\times n},\quad V \in \R^{m\times m}$
\item $U,V$ are orthogonal
\begin{itemize}
\item ($U,V$ not necessarily the same dimension)
\end{itemize}
\item $\Sigma$ is ``almost diagonal". 
\begin{itemize}
\item All entries off the main diagonal are 0.
\end{itemize}
\end{itemize}
\item (!!) \textit{Every} matrix has a SVD.
\item SVD provides a framework for understanding the properties of $A$.
\item Can use the SVD to create matrices (Pseudo-Inverse, Lec 10).
\end{itemize}
\end{frame}

\begin{frame}
\frametitle{Questions: SVD}
\begin{enumerate}
\item Explain what the SVD says about the action of any matrix $A$.
\medskip

\item Recall from Recitation 3 the proof we used to show\\
\qquad $rank(A) = rank(A^TA)$.\\
 Use the SVD to show that:\\
 \qquad $rank(A)=rank(A^TA)=rank(AA^T)=rank(A^T)$.
\medskip
\end{enumerate}

\end{frame}


\begin{frame}
\frametitle{Solutions 1: SVD}
\begin{enumerate}
\item Explain what the SVD says about the action of any matrix $A$.
\begin{solution}
Let $A$ have SVD $A=U\Sigma V^T$.\\
Transforming any vector $x$ by $A$ is equivalent to transforming it by $V^T$, $\Sigma$ and then $U$. \\
Applying $V^T$ is applying a rotation/flipping.\\
Applying $\Sigma$ scales/squashes each (standard basis) axis, and also transforms $V^Tx$ to a different space.\\
Applying $U$ also applies a rotation/flipping.
\end{solution}
\end{enumerate}
\end{frame}

\begin{frame}
\frametitle{Solutions 2: SVD}
\begin{enumerate}
\item[2.] Recall from Recitation 3 the proof we used to show\\
\qquad $rank(A) = rank(A^TA)$.\\
Use the SVD to show that:\\
\qquad $rank(A)=rank(A^TA)=rank(AA^T)=rank(A^T)$.
\begin{solution}
Recall that if B,C are invertible, then rank(BAC) = rank(A).
$A = U\Sigma V^T$ \\
$A^TA = V\Sigma^T \Sigma V^T$ \\
$AA^T = U\Sigma \Sigma^T U^T$ \\
$A^T = V\Sigma^T U^T$ \\
Based on the definition of $\Sigma$, we can easily do the matrix multiplication, and see
that:\\
\qquad $rank(\Sigma)=rank(\Sigma^T\Sigma)=rank(\Sigma\Sigma^T)=rank(\Sigma^T)$.\\
We can then deduce that:\\
\qquad $rank(A)=rank(A^TA)=rank(AA^T)=rank(A^T)$.
\end{solution}
\end{enumerate}
\end{frame}


\begin{frame}
\frametitle{Questions: More SVD}
\smallskip
\begin{enumerate}
\item Give basis transformations for $U$, $\Sigma$, $V^T$.\\
(Your answer should look like $T(a) = b$. where you select convenient $a$'s that form a basis for the origin space of T.  Be careful about the dimensions w/ $\Sigma$!)

\medskip
\item Let $x\in \R^m$ s.t $x = \sum_{i=1}^m\alpha_i v_i$, where $v_i$ are the columns of $V$.\\
 and write the expressions for $V^Tx$, $\Sigma V^Tx$, and $U\Sigma V^Tx$.
\medskip
\item Which vectors span the $Im(A)$? Which vectors span the $Ker(A)$? Which vectors span $Im(A^T)$? Which vectors span $Ker(A^T)$?
\end{enumerate}
\end{frame}


\begin{frame}
\frametitle{Solutions 1: More SVD}
Let $A\in \R^{n\times m}$, where $n>m$, have SVD $A=U\Sigma V^T$.\\ 
Let $e_{i,m}, e_{i,n}$ denote the $i$th standard basis vector in $\R^m$, and $\R^n$.
\begin{enumerate}
\item Give basis transformations for $U$, $\Sigma$, $V^T$.\\
(Your answer should look like $T(a) = b$. where you select convenient $a$'s that form a basis for the origin space of T.  Be careful about the dimensions w/ $\Sigma$!)
\begin{solution}
$V^T(v_i) = e_{i,m}$ \quad for $i\in\{1,...,m\}$ \\
$\Sigma(e_{i,m}) = \sigma_i e_{i,n}$ \quad for $i \in \{1, ..., m\}$ (Note the $m$ here) \\
$U(e_{i,n}) = u_{i}$ \quad for $i \in \{1, ..., n\}$ \\
\end{solution}
\end{enumerate}
\end{frame}



\begin{frame}
\frametitle{Solutions 2: More SVD}
Let $A\in \R^{n\times m}$, where $n>m$, have SVD $A=U\Sigma V^T$.\\ 
Let $e_{i,m}, e_{i,n}$ denote the $i$th standard basis vector in $\R^m$, and $\R^n$.
\begin{enumerate}
\item[2.] Let $x\in \R^m$ s.t $x = \sum_{i=1}^m\alpha_i v_i$, where $v_i$ are the columns of $V$.\\ Write the expressions for $V^Tx$ $\Sigma V^Tx$, and $U\Sigma V^Tx$.
\begin{solution}
Let $e_{i,m}$ denote the $i$th standard basis vector in $\R^m$. \\
$x = \sum_{i=1}^m\alpha_i v_i$\\
$V^Tx = \sum_{i=1}^m\alpha_i e_{i,m}$ \\
$\Sigma V^Tx = \sum_{i=1}^{min(m,n)} \sigma_i \alpha_i e_{i,n}$ \\
$U \Sigma V^Tx = \sum_{i=1}^{min(m,n)} \sigma_i \alpha_i u_i$ \\
\end{solution}
\end{enumerate}
\end{frame}

\begin{frame}
\frametitle{Solutions 3: More SVD}
Let $A\in \R^{n\times m}$, where $n>m$, have SVD $A=U\Sigma V^T$.\\ 
Let $e_{i,m}, e_{i,n}$ denote the $i$th standard basis vector in $\R^m$, and $\R^n$.
\begin{enumerate}
\item[3.] Which vectors span the $Im(A)$? Which vectors span the $Ker(A)$? Which vectors span $Im(A^T)$? Which vectors span $Ker(A^T)$?
\begin{solution}
Let $\sigma_1,...,\sigma_k > 0$ , and $\sigma_{k+1},...,\sigma_{min(m,n)} = 0$.\\
Let $u_1,...,u_n$ be the columns of $U$ and $v_1,...,v_m$ be the columns of $V$.\\
\medskip

$u_1,...,u_k$ span the $Im(A)$.\\
$v_{k+1},..., v_m$ span $Ker(A)$. \\
\medskip
$v_1,...,v_k$ span $Im(A^T)$. \\
$u_{k+1},...,u_n$ span $Ker(A^T)$.\\
\end{solution}
Note the $u$'s and $v$'s!
\end{enumerate}
\end{frame}

\begin{frame}
\frametitle{Questions: SVD, Midterm, Regression}
Midterm Q6 using SVD
Let $M\in \R^{n\times m}$. Let $n>m$, and $M$ have full rank. Let $M$ have SVD $M= U\Sigma V^T$.
\begin{enumerate}

\item Show that $M^TM$ is invertible.
\item Which vectors span the $Im(M)$? Write the matrix of orthogonal projection onto $Im(M)$ and give a basis transformation for that matrix.
\item Let $w\in \R^n$, and $u$ be the orthgonal projection of $w$ onto $Im(M)$. Show that $M^Tu = M^Tw$.
\item Show that $M(M^TM)^{-1}M^T$ is the matrix of an orthogonal projection onto $Im(M)$.
\end{enumerate}
\end{frame}

\begin{frame}
\frametitle{Solutions 1: SVD, Midterm, Regression}
Midterm Q6 using SVD
Let $M\in \R^{n\times m}$. Let $n>m$, and $M$ have full rank. Let $M$ have SVD $M= U\Sigma V^T$.
\begin{solution}
\begin{enumerate}
\item Show that $M^TM$ is invertible.\\
``Questions: SVD" Question 2 earlier in recitation. \\
M is full rank in this case.
\medskip
\item Which vectors span the $Im(M)$? Write the matrix of orthogonal projection onto $Im(M)$ and give a basis transformation for that matrix.\\
``Questions: More SVD" Question 3 earlier in recitation.\\
$u_1,...,u_m$ span $Im(M)$.
 $P_{Im(M)} = U_rU_r^T$ where:\\
${\fonteight U_r = \begin{bmatrix}
\vert &        & \vert \\
u_1   & \hdots & u_m \\
\vert &        & \vert \\
\end{bmatrix}
}$\\
$U_rU_r^T(u_i) = u_i$ for $i\in \{1,...,m\}$.\\
$U_rU_r^T(u_i) = 0$ for $i\in \{m+1,...,n\}$.

\end{enumerate}
\end{solution}
\end{frame}

\begin{frame}
\frametitle{Solutions 2: SVD, Midterm, Regression}
Midterm Q6 using SVD
Let $M\in \R^{n\times m}$. Let $n>m$, and $M$ have full rank. Let $M$ have SVD $M= U\Sigma V^T$.
\begin{enumerate}
\item[3.] Let $w\in \R^n$, and $u$ be the orthgonal projection of $w$ onto $Im(M)$. Show that $M^Tu = M^Tw$.\\
\begin{solution}

\medskip
Let $w = \sum_{i=1}^n \alpha_i u_i$.\\
$u = P_{Im(M)}(w)$, so \\
\quad $u = U_rU_r^Tw$, and\\
\quad $u = \sum_{i=1}^m \alpha_i u_i$. (note change in summation).\\
Now,\\
\quad $M^Tu = \sum_{i=1}^{m} \sigma_i \alpha_i v_i$, and\\
\quad $M^Tw = \sum_{i=1}^{min(m,n)} \sigma_i \alpha_i v_i$ \quad (From More SVD Q2) \\
Since $n>m$, $M^Tu = M^Tw$.
\item[4.] Show that $M(M^TM)^{-1}M^T$ is the matrix of an orthogonal projection onto $Im(M)$.
\end{solution}
\end{enumerate}

\end{frame}

\begin{frame}
\frametitle{Solutions 2: SVD, Midterm, Regression}
Midterm Q6 using SVD
Let $M\in \R^{n\times m}$. Let $n>m$, and $M$ have full rank. Let $M$ have SVD $M= U\Sigma V^T$.
\begin{enumerate}
\item[4.] Show that $M(M^TM)^{-1}M^T$ is the matrix of an orthogonal projection onto $Im(M)$.
\begin{solution}
First, note that $(M^TM)^{-1} = (V \Sigma^T \Sigma V^T)$, and that $\Sigma^T \Sigma$ has rank $m$ and is invertible, where the diagonal entries of $(\Sigma^T \Sigma)^{-1}$ are reciprocals of the entries in $(\Sigma^T \Sigma)$. \\
Now,  \\
\quad $M(M^TM)^{-1}M^T = (U \Sigma V^T)(V (\Sigma^T \Sigma)^{-1} V^T)(V \Sigma^T U^T) $ \\ 
\quad $M(M^TM)^{-1}M^T = U \Sigma (\Sigma^T \Sigma)^{-1} \Sigma^T U^T $ \\ 
Doing the matrix multiplication of  $\Sigma (\Sigma^T \Sigma)^{-1} \Sigma^T $ gives \\

\quad $\Sigma (\Sigma^T \Sigma)^{-1}) \Sigma^T = I_{m,n}$,\\
where $I_{m,n}$ is the $n\times n$ matrix where the first $m$ diagonal entries are $1$. \\
Finally, its easy to show that $UI_{m,n}U^T = U_rU_r^T$.


\end{solution}
\end{enumerate}

\end{frame}

\begin{frame}
\frametitle{Concluding Remarks}
\begin{itemize}
\item Throughout the course, we've emphasized viewing problems from different points of view
\item Matrix multiplication
\begin{itemize}
\item Inner product interpretation
\item Linear combination of columns interpretation
\end{itemize}

\item Linear transformations
\begin{itemize}
\item as matrices (matrix mechanics)
\item as letters (transformations on basis vectors)
\end{itemize}
\item This recitation really emphasizes fluid switching between all of these frameworks
\item All the things we've covered are deeply connected to eachother.
\end{itemize}
\end{frame}
\end{document}
