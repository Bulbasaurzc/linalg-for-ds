\documentclass{beamer}
\usepackage{amsthm}
\usepackage[utf8]{inputenc}
\usetheme{Madrid}
\usepackage{../latex_style/packages_old}
\usepackage{../latex_style/notations_old}
\usepackage{outlines}
\usepackage{enumitem}
\renewenvironment{itemize}
\usefonttheme{serif}
\def\labelenumi{\theenumi}
\usefonttheme{serif}

%\setbeamertemplate{itemize items}[default]
%\setbeamertemplate{enumerate items}[default]

% define smaller font command
\newcommand*{\horzbar}{\rule[.5ex]{2.5ex}{0.5pt}}
\newcommand\fonteight{\fontsize{8}{9.6}\selectfont}
\newcommand\fontten{\fontsize{10}{1.2}\selectfont}

%define enumerate with periods
\renewenvironment{enumerate}%
{\begin{list}{\arabic{enumi}.}%  \langle ------ dot here
      {\setlength{\leftmargin}{2.5em}%
       \setlength{\itemsep}{-\parsep}%
       \setlength{\topsep}{-\parskip}%%
       \usecounter{enumi}}%
 }{\end{list}}
%define itemize with arrows
\renewenvironment{itemize}%
{\begin{list}{$\blacktriangleright$}%  \langle ------ dot here
      {\setlength{\leftmargin}{2.5em}%
       \setlength{\itemsep}{-\parsep}%
       \setlength{\topsep}{-\parskip}%%
       \usecounter{enumi}}%
 }{\end{list}}

%Information to be included in the title page:
\title{Recitation 4}
\author{Alex Dong}
\institute{CDS, NYU}
\date{Fall 2020}


\makeatletter
\setbeamertemplate{navigation symbols}{}
\setbeamertemplate{footline}{
  \leavevmode%
  \hbox{%
  \begin{beamercolorbox}[wd=.4\paperwidth,ht=2.25ex,dp=1ex,center]{author in head/foot}%
    \usebeamerfont{author in head/foot}\insertshortauthor\expandafter\ifblank\expandafter{\beamer@shortinstitute}{}{~~(\insertshortinstitute)}
  \end{beamercolorbox}%
  \begin{beamercolorbox}[wd=.3\paperwidth,ht=2.25ex,dp=1ex,center]{title in head/foot}%
    \usebeamerfont{title in head/foot}\insertshorttitle
  \end{beamercolorbox}%
  \begin{beamercolorbox}[wd=.3\paperwidth,ht=2.25ex,dp=1ex,right]{date in head/foot}%
    \usebeamerfont{date in head/foot}\insertshortdate{}\hspace*{2em}
    \insertframenumber{} / \inserttotalframenumber\hspace*{2ex} 
  \end{beamercolorbox}}%
  \vskip0pt%
}
\setbeamertemplate{navigation symbols}{}
\makeatother

\begin{document}
%1
\frame{\titlepage} 
%2

\begin{frame}
\frametitle{Norms}
\begin{itemize}
\item Norms measure distances!
\item Think about all the ``natural" properties of distance that make sense.
\begin{itemize}
\item distance = 0 means at the same point
\item distance is always non-negative
\item distance follows triangle inequality (at least in Euclidean space)
\end{itemize}
\end{itemize}

\end{frame}

\begin{frame}
\frametitle{Norms}
Shorthand way to remember what the properties do.
\begin{definition}[Norm]
	A norm $\| \cdot \|$ on $V$ verifies the following points:
	\begin{enumerate}
		\item \emph{Triangular inequality}: $\|u + v\| \leq \|u\| + \|v\|$ \hfill ``Euclidean space"
		\item \emph{Homogeneity}: $\| \alpha v \| = |\alpha|\times \| v\|$ \hfill ``farther actually means farther"
		\item \emph{Positive definiteness}: if $\|v\| = 0 \implies v=0$. \hfill ``Non-negative"
	\end{enumerate}
\end{definition}
\end{frame}

\begin{frame}
\frametitle{Inner Products}
\begin{definition}[Inner product]
	Let $V$ be a vector space.
	An inner product on $V$ is a function $\langle \cdot, \cdot \rangle$ from $V \times V$ to $\R$ that verifies the following points:
	\begin{enumerate}
		\item \emph{Symmetry}: $\langle u, v \rangle = \langle v, u\rangle$ for all $u,v \in V$.
		\item \emph{Linearity}: $\langle u+v, w \rangle = \langle u, w\rangle + \langle v, w\rangle$ and $\langle \alpha v, w \rangle = \alpha \langle v, w \rangle$ for all $u,v,w \in V$ and $\alpha \in \R$.
		\item \emph{Positive definiteness}: $\langle v, v\rangle \geq 0$ with equality if and only if $v = 0$.
\end{enumerate}
\end{definition}
\begin{itemize}
\item Definition of inner product does not reveal it's purpose.
\item \textbf{In this class, we always use the Euclidean inner product.}
\begin{itemize}
\item $\langle u,v \rangle = u^Tv$
\end{itemize}
\item (!!) Inner products are (indirectly) used for a notion of angles.
\item $cos(\theta) = \frac{ \langle u,v \rangle }{\|u\| \|v\|}$
\end{itemize}
\end{frame}

\begin{frame}
\frametitle{Inner Products in Machine Learning (\&)}
\begin{itemize}
\item Inner products can be used as a measure of similarity
\item Kernel Tricks (\&) - Increase Data Complexity
\begin{itemize}
\item Sometimes you have to calculate  $x_{old}^T x_{new}$, equivalently $ \langle x_{old},x_{new} \rangle $
\item You can replace the inner product with a inner product in a higher dimensional space
\item Instead of calculating $ \langle x_{old},x_{new} \rangle $, define a function $K$ and calculate $ \langle K(x_{old}),K(x_{new}) \rangle $
\item If you pick ``the right" higher dimensional space, your data can be a lot easier to work with
\end{itemize}
\end{itemize}
\footnotetext{(\&) denotes extra material not covered in this course}

\end{frame}


\begin{frame}
\frametitle{Questions 1: Norms and Inner Products}

\begin{enumerate}
\item[1.] Which of the following functions are inner products for $x,y\in\R^3$?
\begin{enumerate}
\item[i.] $f(x,y) = x_1y_2+x_2y_3+x_3y_1$
\item[ii.] $f(x,y) = x_1^2y_1^2+x_2^2y_2^2+x_1^2y_1^2$
\item[iii.] $f(x,y) = x_1y_1+x_3y_3$
\end{enumerate}
\item[2.] For $A \in \R^{m\times n}$ and $x\in \R^n$, prove that
$$ \|Ax\| \leq \|x\|\sqrt{\sum_{i=1}^m \sum_{j=1}^n} A_{i,j}^2$$
\end{enumerate}

\end{frame}

\begin{frame}
\frametitle{Solutions 1: Norms and Inner Products}

\begin{enumerate}
\item[1.] Which of the following functions are inner products for $x,y\in\R^3$?
\begin{solution}
\begin{enumerate}
\item[i.] $f(x,y) = x_1y_2+x_2y_3+x_3y_1$ \hfill False\\
Consider $u=[1,0,0]^T$ and $v=[0,1,0]^T$.\\
$ \langle u,v \rangle =1$, but $ \langle v,u \rangle  = 0$. (Not symmetric)
\medskip
\item[ii.] $f(x,y) = x_1^2y_1^2+x_2^2y_2^2+x_3^2y_3^2$ \hfill False\\
Consider $v = [1,0,0]^T.$ \\
$ \langle 2v,v \rangle  = 4$, but $2 \langle v,v \rangle  = 2$. (Not linear)
\medskip
\item[iii.] $f(x,y) = x_1y_1+x_3y_3$ \hfill False \\
Consider $v = [0,1,0]^T$. \\
 $ \langle v,v \rangle  = 0$, but $v\neq0$. (Not positive definite)
\end{enumerate}
\end{solution}
\end{enumerate}
\end{frame}


\begin{frame}
\frametitle{Solutions 1: Norms and Inner Products}

\begin{enumerate}
\item[2.] For $A \in \R^{m\times n}$ and $x\in \R^n$, prove that
$$ \|Ax\| \leq \|x\|\sqrt{\sum_{i=1}^m \sum_{j=1}^n} A_{i,j}^2$$
\begin{solution}
Let $A = {\fonteight\begin{bmatrix}
			\horzbar    & \mathbf{a_1}^T    & \horzbar  \\
			\vdots    & \vdots & \vdots  \\
			\horzbar    & \mathbf{a_m}^T    & \horzbar   \\
		 \end{bmatrix}}$
	and
	$x =  \begin{bmatrix}
			\vline \\
			 x \\   
			\vline \\ 
		 \end{bmatrix}$ .
Observe that 
$Ax =  \begin{bmatrix}
			 \langle \mathbf{a_1},x \rangle \\
			 \vdots \\   
			\langle \mathbf{a_m},x \rangle \\ 
		 \end{bmatrix}$ .
Now,\\
\qquad $\|Ax\|^2 = \sum_{i=1}^m  |\langle \mathbf{a_i},x \rangle ^2|$ \qquad by definition of norm \\
\qquad $\|Ax\|^2 \leq \sum_{i=1}^m \|\mathbf{a_i}\|^2 \|x\|^2$ \qquad by Cauchy-Schwarz \\
\qquad $\|Ax\| \leq (\sum_{i=1}^m \|\mathbf{a_i}\|^2 \|x\|^2)^{.5}$\\
\qquad $\|Ax\| \leq \|x\|(\sum_{i=1}^m \|\mathbf{a_i}\|^2 )^{.5}$ \\
\qquad $\|Ax\| \leq \|x\|(\sum_{i=1}^m \sum_{j=1}^n A_{i,j})^{.5}$ \qquad by definition of $\mathbf{a_i}$

\end{solution}
\end{enumerate}
\end{frame}

\begin{frame}
\frametitle{Orthogonality}
\begin{itemize}
\item Angles can be used as a measure of similarity
\item Vectors $u,v$ are orthogonal if and only if $ \langle u,v \rangle =0$
\item Vectors are orthogonal $\implies$ vectors are as dissimilar as possible
\item Orthogonal coordinate systems are nice because we can view each coordinate ``independently'' (we will prove later).
\item Gram-Schmidt Process (Lec 5) allows us to change any basis into an orthonormal basis.
\end{itemize}
\end{frame}


\begin{frame}
\frametitle{Orthogonal Projections}
\begin{itemize}
\item Projections form an important part of linear algebra.
\begin{itemize}
\item We can view the action of a matrix and how it affects a certain subspace
\item We can simplify our data by picking the subspace ``closest'' to the data (PCA, Lec 7)
\item We can find the best-fit line/plane/subspace  (Linear regression, Lec 9)
\end{itemize}
\medskip
\item \textit{Orthogonal} projections are a special kind of projection
\begin{itemize}
\item They preserve the original vector components (in the orthogonal basis)

\end{itemize}
\end{itemize}

\end{frame}



\begin{frame}

\frametitle{Questions: Orthogonality}
\begin{enumerate}
\item Let $v_1,...,v_k$ be a list of orthogonal vectors. Show that $v_1,...,v_k$ are linearly independent.
\item Let U be the subspace of $\R^n$ with orthonormal basis $u_1,...,u_k$.
\begin{enumerate}
 \item[i.] Prove that the orthogonal projection of $v \in \R^n$ onto $U$ can be expressed as $P_U  = \sum_{i=0}^k  \langle v,u_i \rangle u_i$. (Use the fact that the orthonormal basis for a subspace of $\R$ can be extended to obtain an orthonormal basis for $\R$)
 \item[ii.] Prove that $P_U(v)\leq \|v\|$
 \item[iii.] Prove that $v-P_U(v)$ is orthogonal to $P_U(v)$
\end{enumerate}
\end{enumerate}
\end{frame}

\begin{frame}
\frametitle{Solutions: Orthogonality}

\begin{solution}
\begin{enumerate}
\item Let $v_1,...,v_k$ be a list of non-zero orthogonal vectors. Show that $v_1,...,v_k$ are linearly independent.\\ \medskip
\fonteight
Let $\alpha_1,...,\alpha_k \in \R$ s.t $\sum_{i=1}^k \alpha_i v_i = \vec{0}.$\\

Consider $ \langle \sum_{i=1}^k \alpha_i v_i, \sum_{j=1}^k \alpha_j v_j \rangle .$
\begin{align*}
0 &=\  \langle \vec{0},\vec{0} \rangle \ \\
&=\  \langle \sum_{i=1}^k \alpha_i v_i, \sum_{j=1}^k \alpha_j v_j \rangle  \\
&=\sum_{i=1}^k \alpha_i^2  \langle v_i,v_i \rangle , \sum_{i\neq j} \alpha_i \alpha_j  \langle v_i, v_j \rangle  \\
0&= \sum_{i=1}^k \alpha_i^2 \qquad \text{ by orthonormality of }v_i, v_j
\end{align*}\\
So $\alpha_1,...,\alpha_k = 0$.
\end{enumerate}
\end{solution}
\end{frame}



\setlength{\abovedisplayskip}{0pt}%
\setlength{\belowdisplayskip}{0pt}%
\setlength{\abovedisplayshortskip}{0pt}%
\setlength{\belowdisplayshortskip}{0pt}%
\setlength{\jot}{0pt}
\begin{frame}
\frametitle{Solutions: Orthogonality}

\begin{solution}
\fonteight
Let U be the subspace of $\R^n$ with orthonormal basis $u_1,...,u_k$.
\begin{enumerate}
\item[2i.] Prove that the orthogonal projection of $v \in \R^n$ onto $U$ can be expressed as $P_U(v)  = \sum_{i=0}^k  \langle v,u_i \rangle u_i$. \\

Let $u_{k+1},...,u_n$ be an orthonormal basis extension for $ u_1,...,u_k$.\\
Then $u_1,...,u_k,u_{k+1},...,u_n$ form an orthonormal basis for $\R^n$\\
Now, let $v = \sum_{i=1}^n \alpha_i u_i$ where $\alpha_i=\  \langle v,u_i \rangle $ and let $x\in U$, where $x=\sum_{j=1}^k  \beta_i u_i$.\\
We want to find $\argmin_{x\in U} \|v-x\|$.
\begin{align*}
 \|v-x\| &= \|\ \sum_{i=1}^n \alpha_i u_i - \sum_{j=1}^k  \beta_i u_i \| \\
         &= \|\ \sum_{j=1}^k (\alpha_i-\beta_i) u_i - \sum_{i=k+1}^n  \alpha_i u_i \| \\
         &= \sqrt{\sum_{j=1}^k (\alpha_i-\beta_i)^2 + \sum_{i=k+1}^n  \alpha_i^2} \qquad \text{by orthonormality}
\end{align*}
$\|v-x\|$ is minimized when $\alpha_i = \beta_i \quad \forall i \in \{1,...,k\}$\\
This implies that $\beta_i =  \langle v,u_i \rangle $.\\
 So $P_U(v)=argmin_{x\in U}\|v-x\| = \sum_{i=0}^k  \langle v,u_i \rangle u_i$.
\end{enumerate}
\end{solution}
\end{frame}


\begin{frame}
\frametitle{Solutions: Orthogonality}
\begin{solution}
Let U be the subspace of $\R^n$ with orthonormal basis $u_1,...,u_k$.
\begin{enumerate}
 \item[2ii.] Prove that $P_U(v)\leq \|v\|$\\
$P_U(v) = \sum_{i=1}^k  \langle v,u_i \rangle u_i$ from $2i$
\begin{align*}
\|P_U(v)\|^2 &= \| \sum_{i=1}^k  \langle v,u_i \rangle u_i \|^2  \\
             &= \sum_{i=1}^k  \| \langle v,u_i \rangle u_i\|^2 \qquad \text{ by Pythagorean Theorem} \\
             &\leq \sum_{i=1}^n \| \langle v,u_i \rangle u_i\|^2 \qquad \text {add extra components}\\
             &= \|\sum_{i=1}^n  \langle v,u_i \rangle u_i\|^2 \qquad \text {Pythagorean Theorem} \\ 
             &= \|v\|^2
\end{align*}
So $P_U(v)\leq \|v\|$ 
\end{enumerate}
\end{solution}
\end{frame}

\begin{frame}
\frametitle{Solutions: Orthogonality}
\begin{solution}
Let U be the subspace of $\R^n$ with orthonormal basis $u_1,...,u_k$.
\begin{enumerate}
\item[2iii.] Prove that $v-P_U(v)$ is orthogonal to $P_U(v)$
$P_U(v) = \sum_{i=1}^k  \langle v,u_i \rangle u_i$ \qquad from $2i$\\
$v = \sum_{i=0}^n  \langle v,u_i \rangle u_i$ \qquad since $u_1,...,u_n$ is a orthonormal basis.
\begin{align*}
v-P_U(v) &= \sum_{i=1}^n  \langle v,u_i \rangle u_i - \sum_{i=1}^k \langle v,u_i \rangle u_i\\
&= \sum_{i=k+1}^n \langle v,u_i \rangle u_i\\
\langle v-P_U(v),v \rangle \ &=\ \langle (\sum_{i=k+1}^n \langle v,u_i \rangle u_i),(\sum_{i=1}^k \langle v,u_i \rangle u_i) \rangle \\
&=0 \qquad u_i \text{ are pairwise orthogonal}.
\end{align*}
\end{enumerate}
\end{solution}
\end{frame}

\begin{frame}
\frametitle{Questions: Orthogonal Complements}
Let $S, U$ be subspaces of a vector space $V$.\\
Prove the following statement:
\begin{enumerate}

\item $S\subset U \implies S^\perp \supset U^\perp$
\end{enumerate}
\bigskip
Let $A\in\R^{n\times m}$.
Assume the Euclidean inner product.
\begin{enumerate}
\item[2.] (!) Prove that $Im(A^T) = Ker(A)^\perp$.
\end{enumerate}
\bigskip
\qquad (Hint: $\implies$ is easy. Use (1) for $\impliedby$)\\
\end{frame}

\begin{frame}
\frametitle{Solutions: Orthogonal Complements}
\begin{enumerate}
\item $S\subset U \implies S^\perp \supset U^\perp$
\end{enumerate}
\begin{solution}
Let $x\in U^\perp$, and $z\in S$. \\
Since $z\in S$ and $S\subset U$, then $z\in U$.\\
Now, since $x\in U^\perp$ and $z \in U$, then $\langle x,z \rangle$ = 0.\\
So $x \in S^\perp$.
\end{solution}
\end{frame}

\begin{frame}
\frametitle{Solutions: Orthogonal Complements}
\begin{enumerate}
\item[2.] Prove that $Im(A^T) = Ker(A)^\perp$.
\end{enumerate}
\begin{solution}
$\implies$ \\
Let $x\in \Im(A^T)$. Then $\exists y$ s.t $x = A^Ty$. We show $x\in Ker(A)^\perp$.\\
Let $v\in Ker(A)$. Then $Av=0$. \\
Consider $\langle x,v \rangle$. \\
\qquad $\langle x,v \rangle = x^Tv = y^TAv =\langle y,Av \rangle =\langle y,0 \rangle = 0$
Then $x\in Ker(A)^\perp$.
$\impliedby$.\\
We use 1. to show $Im(A^T)^\perp \subset Ker(A)$ instead.\\
Let $x\in Im(A^T)^\perp$. \\
Consider $Ax$. We show $\langle x,A^Ty \rangle = 0$ for all $y\in \R^n$. \\
Since $x\in Im(A^T)^\perp$, then $\forall y\in \Im(A^T)$, $\langle x,y \rangle = x^Ty = 0$.\\
Consider $\|Ax\|$.\\ 
\qquad $\|Ax\| = x^TA^TAx = x(A^TAx)$.\\
Since $A^TAx \in Im(A^T)$, then $\|Ax\| = 0$, so $Ax=0$.\\ 
Now, by 1, we can conclude that $Ker(A)^\perp \subset Im(A^T)$.

\end{solution}
\end{frame}

\begin{frame}
\frametitle{Appendix starts after here}

\end{frame}
\begin{frame}
\frametitle{Idempotence}
Lets take a step back.\medskip
\begin{itemize}
\item $P_S$ is an \textit{orthogonal} projection $\iff$ $P_S = VV^T$
\begin{itemize}
 \item V has orthonormal columns that form a basis for $S$.\\
\end{itemize}
\item There is a more general definition of a projection - known as \textit{idempotence}.

\begin{definition}[Idempotence]
An matrix P is idempotent when $P^2=P$.
\end{definition}
\end{itemize}
An idempotent matrix is also called a \textit{projection} or \textit{projection matrix}.
\end{frame}

\begin{frame}
\frametitle{Questions: Orthogonal Projections vs Idempotence}
\begin{definition}[Idempotence]
An matrix P is idempotent when $P^2=P$.
\end{definition}
\bigskip
\begin{enumerate}
\item Show that $X(X^TX)^{-1}X^T$ is idempotent.
\item Show that all orthogonal projections are idempotent.
\item Give an example of an idempotent matrix that is not an orthogonal projection.  \\
(Hint: Show that your matrix does not minimize the distance to subspace it projects onto.)
\end{enumerate}
\end{frame}

\begin{frame}
\frametitle{Solutions: Orthogonal Projections vs Idempotence}
\begin{solution}
\begin{enumerate}
\item Show that $X(X^TX)^{-1}X^T$ is idempotent. \\
\begin{align*}
P^2 &= (X(X^TX)^{-1}X^T)(X(X^TX)^{-1}X^T) \\
&= X(X^TX)^{-1}(X^TX)(X^TX)^{-1}X^T \\
&= X(X^TX)^{-1}X^T
\end{align*}
\item Show that all orthogonal projections are idempotent.\\
\medskip
Let $P$ be an orthogonal projection.\\
Recall that all orthogonal projections take the form $VV^T$, where $V\in \R^{n\times k}$ has orthonormal columns. \\
Note that $V^TV = I_k$, the identity matrix in $\R^{k\times k}$. \\
Then $P^2 = (VV^T)(VV^T) = V(V^TV)V^T = VI_kV^T = VV^T = P$

\end{enumerate}
\end{solution}
\end{frame}

\begin{frame}
\frametitle{Solutions: Orthogonal Projections vs Idempotence}
\begin{solution}
\begin{enumerate}
\item[3.] Give an example of an idempotent matrix that is not an orthogonal projection.  \\
\fonteight

Consider the matrix 
$A = \begin{bmatrix}
			1   & 1   \\
			0   & 0\\
		 \end{bmatrix}$\\
It's easy to see $A^2 = A$, and
$Im(A) = \{\begin{bmatrix}
x  \\
0  \\
\end{bmatrix}\ | \ x\in \R\}$\\
Consider the vector $v = \begin{bmatrix}
			2\\
			2\\
		 \end{bmatrix}$\\
The closest vector in Im(A) is $v_{Im(A)} = \begin{bmatrix}
			2\\
			0\\
		 \end{bmatrix}$, but $Av=\begin{bmatrix}
			4\\
			0\\
		 \end{bmatrix}$\\
\medskip
Note: Rigorously speaking, we need to prove that $v_{Im(A)} = \begin{bmatrix}
			2\\
			0\\
		 \end{bmatrix}$

is the closest vector in $Im(A)$. We can do this by constructing an orthogonal projection onto $Im(A)$,
which is found by setting $V =\begin{bmatrix}
			1\\
			0\\
\end{bmatrix}$, and calculating $VV^T = \begin{bmatrix}
			1   & 0   \\
			0   & 0\\
		 \end{bmatrix}$\\
\end{enumerate}
\end{solution}
\end{frame}


\end{document}