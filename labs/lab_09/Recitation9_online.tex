\documentclass{beamer}

\usepackage{../../latex_style/beamerthemeExecushares}
\usepackage{../../latex_style/notations}
\usepackage{dsfont}

\title{Recitation 9}
\author{Carles Domingo}
\date{Fall 2020}


\begin{document}

\frame{\titlepage} 

\setcounter{showProgressBar}{0}
\setcounter{showSlideNumbers}{1}

\begin{frame}[t]
\frametitle{Adjacency matrix and Laplacian}
\begin{definition}[Adjacency matrix]
We define the adjacency matrix $A \in \R^{n \times n}$ of a graph $G$ with $n$ nodes as
\begin{equation*} 
A_{ij} = \begin{cases}
1 &\text{if } i \sim j \\
0 &\text{otherwise}
\end{cases}
\end{equation*}
\end{definition}

\begin{definition}[Laplacian of a matrix]
The Laplacian matrix of $G$ is defined as
\begin{equation*} 
L = D - A
\end{equation*}
where $D$ is the degree matrix $D = \text{diag}(\text{deg}(1), \dots, \text{deg}(n))$.
\end{definition}
Remember that $A$ is symmetric and positive semidefinite, and 1 is an eigenvalue of $A$ with eigenvector $\mathds{1}$.
\end{frame}

\begin{frame}[t]
\frametitle{Laplacian}
The Fiedler eigenvalue $\lambda_2$ is the smallest eigenvalue of $L$ larger than 0.
\begin{enumerate}
\item Show that when we add an edge to the graph, the Fiedler eigenvalue increases.
\end{enumerate}
\pause
\end{frame}

\begin{frame}[t]
\frametitle{Normalized Laplacian}
\vspace{-10pt}
A bipartite graph is a graph such that the set of nodes can be split into two subsets such that all the edges of the graph are between nodes in different subsets.

2. Show that the largest eigenvalue of $L_{norm} = D^{-1/2} L D^{-1/2}$ is less or equal than 2.

3. Show that the largest eigenvalue of $L_{norm}$ is equal to 2 if and only if the graph has some bipartite connected component.
\pause
\pause
\end{frame}

\begin{frame}[t]
\frametitle{Adjacency matrix}
Show that the largest eigenvalue $\lambda_n$ of the adjacency matrix $A$ is larger or equal than the average of the degrees of the nodes and smaller or equal than the maximum degree. 
\pause
\end{frame}

\begin{frame}[t]
\frametitle{Complete graphs}
\begin{enumerate}
\item What are the eigenvalues of the Laplacian of the complete graph over $n$ nodes?
\end{enumerate}
\pause
\end{frame}

\begin{frame}[t]
\frametitle{Spectral clustering}
Reminder of the spectral clustering algorithm:

Input: Graph Laplacian $G$, number of clusters $k$.
\begin{enumerate}
\item Compute the first $k$ eigenvectors of the graph Laplacian.
\item Associate to each node $i$ the vector $x_i = (v_2(i), \dots, v_k(i))$.
\item Cluster the points $x_1, \dots, x_n$ with $k$-means, for example.
\end{enumerate}
\end{frame}

\begin{frame}[t]
\frametitle{Extra SVD Question}
Midterm 2019 Q6: Let $M\in \R^{n\times m}$. Let $n \geq m$, and $M$ have full rank. Let $M$ have SVD $M= U\Sigma V^T$.
\begin{enumerate}
\item Show that $M^TM$ is invertible.
\item Which vectors span the $Im(M)$? Write the matrix of orthogonal projection onto $Im(M)$ and give a basis transformation for that matrix.
\item Let $w\in \R^n$, and $u$ be the orthogonal projection of $w$ onto $Im(M)$. Show that $M^Tu = M^Tw$.
\item Show that $M(M^TM)^{-1} M^T$ is the matrix of an orthogonal projection onto $Im(M)$.
\end{enumerate}
\end{frame}

\begin{frame}[t]
\frametitle{Extra SVD Question}
\pause
\pause
\pause
\pause
\end{frame}

\begin{frame}[t]
\pause
\end{frame}


\end{document}


