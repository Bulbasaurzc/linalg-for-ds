\documentclass{beamer}
\usepackage{amsthm}
\usepackage[utf8]{inputenc}
\usetheme{Madrid}
\usepackage{../latex_style/packages_old}
\usepackage{../latex_style/notations_old}
\usepackage{outlines}
\usepackage{enumitem}
\renewenvironment{itemize}
\usefonttheme{serif}
\def\labelenumi{\theenumi}
\usefonttheme{serif}
\usepackage{comment}

%\setbeamertemplate{itemize items}[default]
%\setbeamertemplate{enumerate items}[default]

% define smaller font command
\newcommand*{\horzbar}{\rule[.5ex]{2.5ex}{0.5pt}}
\newcommand\fonteight{\fontsize{8}{9.6}\selectfont}
\newcommand\fontten{\fontsize{10}{1.2}\selectfont}

%define enumerate with periods
\renewenvironment{enumerate}%
{\begin{list}{\arabic{enumi}.}%  \langle ------ dot here
      {\setlength{\leftmargin}{2.5em}%
       \setlength{\itemsep}{-\parsep}%
       \setlength{\topsep}{-\parskip}%%
       \usecounter{enumi}}%
 }{\end{list}}
%define itemize with arrows
\renewenvironment{itemize}%
{\begin{list}{$\blacktriangleright$}%  \langle ------ dot here
      {\setlength{\leftmargin}{2.5em}%
       \setlength{\itemsep}{-\parsep}%
       \setlength{\topsep}{-\parskip}%%
       \usecounter{enumi}}%
 }{\end{list}}

%Information to be included in the title page:
\title{Recitation 9}
\author{Alex Dong}
\institute{CDS, NYU}
\date{Fall 2020}


\makeatletter
\setbeamertemplate{navigation symbols}{}
\setbeamertemplate{footline}{
  \leavevmode%
  \hbox{%
  \begin{beamercolorbox}[wd=.4\paperwidth,ht=2.25ex,dp=1ex,center]{author in head/foot}%
    \usebeamerfont{author in head/foot}\insertshortauthor\expandafter\ifblank\expandafter{\beamer@shortinstitute}{}{~~(\insertshortinstitute)}
  \end{beamercolorbox}%
  \begin{beamercolorbox}[wd=.3\paperwidth,ht=2.25ex,dp=1ex,center]{title in head/foot}%
    \usebeamerfont{title in head/foot}\insertshorttitle
  \end{beamercolorbox}%
  \begin{beamercolorbox}[wd=.3\paperwidth,ht=2.25ex,dp=1ex,right]{date in head/foot}%
    \usebeamerfont{date in head/foot}\insertshortdate{}\hspace*{2em}
    \insertframenumber{} / \inserttotalframenumber\hspace*{2ex} 
  \end{beamercolorbox}}%
  \vskip0pt%
}
\setbeamertemplate{navigation symbols}{}
\makeatother
\newcommand{\x}{\vec{x} } 
\newcommand{\y}{\vec{y} } 

\begin{document}
%1
\frame{\titlepage} 
%2

\begin{frame}
\frametitle{Convexity and Optimization}
\begin{itemize}
\item We are temporarily stepping away from \textit{Linearity}
\item Two types of convexity
\begin{itemize}
\item Convexity for functions (we care about this)
\item Convexity for sets
\item They are related! (see epigraph question)
\end{itemize}
\item Convexity implies if a min exists, it must be a global min
\begin{itemize}
\item Optimization is the process to find the minimum
\end{itemize}
\item Convexity is a \textit{global} property
\item Contrast to differentiable, which is a \textit{local} property
\end{itemize}
\end{frame}

\begin{frame}
\frametitle{Questions: Convexity and Epigraph}
Let $f: \R^n \rightarrow \R$.\\
Define the epigraph $epi(f)\subset \R^{n+1}$ to be set of all the points above the graph of $f$:\\
\quad $epi(f) = \{[\x,c] \in \R^{n+1}\ |\ c \geq f(\x) \}$ \\
where [$\bullet,\bullet$] denotes a vector concatentation\\

\begin{enumerate}
\item[0.] For the function  $f: \R^2 \rightarrow \R$, s.t $f(x,y)=x^2+y^2$,\\ draw $epi(f)$
\medskip
\item[1.] Prove that $f$ is convex if and only if $epi(f)$ is convex\\
(Warning, this problem has \textit{really tricky} notation)\\
(Hint $\impliedby$ is easier than $\implies$)
\end{enumerate}
\end{frame}


\begin{frame}
\frametitle{Solutions 1: Convexity and Epigraph}
\begin{solution}
$epi(f)$ is convex $\implies$ $f$ is convex.\\
\medskip 
Assume $epi(f)$ is convex. Let $\x,\y \in \R^n, t\in[0,1]$.\\
Consider $t(f(\x)+(1-t)f(\y)$, \quad convex combo of $f(\vec{x})$,$f(\y)$\\
Since $[\x,f(\x)],[\y,f(\y)]\in epi(f)$, and $epi(f)$ is convex, then,\\
\medskip
\quad $t[\x,f(\x)]+(1-t)[\y,f(\y)] \in epi(f)$ \qquad \quad convexity of $epi(f)$\\
\quad $[t \x +(1-t)\y\ ,\ t f(\x )+(1-t)f(\y )] \in epi(f)$ \qquad add vectors \\
\medskip 
Therefore, by definition of $epi(f)$, we have $\forall x,y\in \R^n$, $t \in [0,1]$ \\
\quad $t f(\x )+(1-t)f(\y ) \geq t \x +(1-t)\y$
So $f$ is convex.

\end{solution}
\end{frame}


\begin{frame}
\frametitle{Solutions 2:  Convexity and Epigraph}
\begin{solution}
$f$ is convex $\implies$ $epi(f)$ is convex.\\
Assume $f$ is convex.\\
 Let $[x,c],[y,d] \in epi(f), t\in[0,1]$. \\
Since $[\x,c],[\y,d)]\in epi(f)$,then we have \\
\quad $c \geq f(\x)$ and $d \geq f(\y)$ \\
and this directly implies\\
\quad $tc \geq tf(\x)$ and $(1-t)d \geq (1-t) f(\y)$ \quad \qquad (*)\\
Now, consider $t[\x,c]+(1-t)[\y,d])$, \quad convex combo of $[\x,c],[\y,d]$ \\
From (*), we have \\
\quad $t[\x,c]+(1-t)[\y,d]) \geq tf(\x) + (1-t) f(\y)$ \\
and since $f$ is convex, then\\
\quad $t[\x,c]+(1-t)[\y,d]) \geq tf(\x) + (1-t) f(\y) \geq f(t\x + (1-t)\y )$\\
\quad $t[\x,c]+(1-t)[\y,d]) \geq f(t\x + (1-t)\y )$ \\
So $t[\x,c]+(1-t)[\y,d])\in epi(f)$.\\
Since this applies $\forall x,y\in \R^n$, $t \in [0,1]$, $epi(f)$ is convex.
\end{solution}
\end{frame}


\begin{frame}

\frametitle{Questions: True and False}
\begin{enumerate}

\item If $f$ has only $1$ global min and no local min, then $f$ is convex
\item Linear combination of two convex functions is convex
\item Convex functions are differentiable at all points
\item Norms are convex functions
\item If $f$ is convex, then g(x) = $f(Ax-b)$ is also convex. ($A\in\R^{n\times n}$ , $b\in \R^n$)
\item Sum of a non-convex function w/ another function can never be convex
\item Union of convex sets is convex
\item Intersection of convex sets is convex
\item Maximum of two convex functions is convex
\item Every subspace is a convex set
\item Every convex set is a subspace
\end{enumerate}
\end{frame}

\begin{frame}
\frametitle{Questions: True and False}
\begin{enumerate}
\item False, $cos(\theta)$, $\theta\in[0,\pi]$
\item False, negative of convex function is not convex
\item False, f(x) = |x|
\item True, Triangle inequality (Prove it!)
\item True, convexity is a global property, so the $Ax-b$ doesnt matter.
\item False sum of f and -f is 0, which is a convex function.
\item False, Easy counter-example
\item True,  True, intersection preserves convexity properties
\item True, Use epigraph proof
\item True, convex combinations are in the subspace by property of subspaces.
\item False, convex sets do not need to contain 0.
\end{enumerate}
\end{frame}


\begin{frame}
\frametitle{Questions: Quadratic Forms}
Let $f: \R^n \rightarrow \R$ be given by $f(x) = x^T Ax$ for some symmetric matrix $A \in\R^{n\times n}$. 
\begin{enumerate}
\item Give conditions on $A$ so that $0$ is the global minimizer of $f$
\end{enumerate}
\end{frame}


\begin{frame}
\frametitle{Solutions: Quadratic Forms}
Let $f: \R^n \rightarrow \R$ be given by $f(x) = x^T Ax$ for some symmetric matrix $A \in\R^{n\times n}$. 
\begin{enumerate}
\item Give conditions on $A$ so that $0$ is the global minimizer of $f$
\begin{solution}
Let $A$ have spectral decomposition $A= U\Lambda U^T$.\\
Let $y\in \R^n$ s.t $y = U^Tx$.\\
Then $f(x)= x^TAx = x^TU\Lambda U^Tx = y^T\Lambda y$, and\\
\quad $y^T\Lambda y = \sum_{i=1}^n \lambda_i y_i^2$\\
 We can see that 0 will be the global minimzer of $f$ if A is positive semi-definite.

\end{solution}
\end{enumerate}
\end{frame}



\end{document}



