\documentclass{beamer}

\usepackage{../../latex_style/beamerthemeExecushares}
\usepackage{../../latex_style/notations}

\title{Lecture 3.1: The rank}
\subtitle{Optimization and Computational Linear Algebra for Data Science}
\author{Léo Miolane}
\date{}

\setcounter{showSlideNumbers}{1}

\begin{document}
\setcounter{showProgressBar}{0}
\setcounter{showSlideNumbers}{0}

\frame{\titlepage}


\setcounter{framenumber}{0}
%\setcounter{showProgressBar}{1}
\setcounter{showSlideNumbers}{1}


\begin{frame}[t]{Rank of a family of vectors}
	\begin{definition}
		We define the rank of a family $x_1, \dots, x_k$ of vectors of $\R^n$ as the dimension of its span:
		$$
		\rank(x_1, \dots, x_k) \defeq \dim (\Span(x_1, \dots, x_k)).
		$$
	\end{definition}
\end{frame}

\begin{frame}[t]{Rank of a matrix}
	\begin{definition}
		Let $M \in \R^{n \times m}$. Let $c_1, \dots, c_m \in \R^n$ be its columns.
		We define
		$$
		\rank(M) \defeq \rank(c_1, \dots, c_m) = \dim(\Im(M)).
		$$
	\end{definition}
\end{frame}

\begin{frame}{Example}
\end{frame}

\begin{frame}[t]{<< Rank of columns = rank of rows >>}
	\begin{block}{Proposition}
		Let $M \in \R^{n \times m}$. Let $r_1, \dots, r_n \in \R^m$ be the rows of $M$ and $c_1, \dots, c_m \in \R^n$ be its columns.
		Then we have
		$$
		\rank(r_1, \dots, r_n) = \rank(c_1, \dots, c_m) = \rank(M).
		$$
	\end{block}
\end{frame}

\begin{frame}[t]{The rank in Data Science}
	\vspace{-0.3cm}
	Consider a matrix $M$ of size $1000 \times 500$:
	$$
	M=
	\begin{pmatrix}
		- & r_1 & - \\
		  & \vdots & \\
		-  & r_{1000} & -
	\end{pmatrix}
	$$
	What does it mean to say that << $\rank(M)=5$ >> ?
\end{frame}
\begin{frame}[t]{The rank in Data Science}
	Imagine now that
	\begin{itemize}
		\item The rows of $M$ corresponds to Netflix's users.
		\item The columns of $M$ corresponds to Netflix's movies.
		\item The entry $M_{i,j}$ is rating of the movie $j$ by the user $i$, assuming that all the users have rated all the movies.
	\end{itemize}
	\vspace{0.5cm}
	\pause

	\begin{center}
	\textbf{Claim:} the rank of $M$ is "small".
	\end{center}
	\vspace{0.5cm}

	\begin{itemize}
		\item The ratings of a user can be obtained as a linear combination of a small number of << profiles >>.
		\item In practice, we do not have access to the full matrix, so we can use this assumption to predict the missing entries.
	\end{itemize}
\end{frame}


\end{document}
