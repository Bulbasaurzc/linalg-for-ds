\documentclass{beamer}

\usepackage{../../latex_style/beamerthemeExecushares}
\usepackage{../../latex_style/notations}

\title{Session 2: Linear transformations and matrices}
\subtitle{Optimization and Computational Linear Algebra for Data Science}
\author{Léo Miolane}
\date{}

\setcounter{showSlideNumbers}{1}

\begin{document}
\setcounter{showProgressBar}{0}
\setcounter{showSlideNumbers}{0}

\frame{\titlepage}

\begin{frame}
	\frametitle{Contents}
	\begin{enumerate}
		\item Recap of the videos
		\item Operation on matrices
		\item Kernel and Image
		\item Why do we care about all these things ? \below{Solving linear systems}
	\end{enumerate}
\end{frame}


\setcounter{framenumber}{0}
\setcounter{showSlideNumbers}{1}

\section{Linear maps \& matrices}

\begin{frame}[t]{Two sides of the same coin}
	\begin{columns}
		\hspace{-0.85cm}
		\begin{column}{0.5\textwidth}
			\vspace{-0.5cm}
			\begin{center}
				\bf Linear map
			\end{center}
			$$
			L: \R^m \to \R^n
			$$
			\vspace{6cm}
		\end{column}
		\vrule
		\begin{column}{0.5\textwidth}
			\vspace{-0.5cm}
			\begin{center}
				\bf Matrix
			\end{center}
			$$
			L \in \R^{n \times m}
			$$
			\vspace{6cm}
		\end{column}
	\end{columns}
	\pause
\end{frame}

\begin{frame}[t]{Rotations in $\R^2$}
	Let $\theta \in \R$.
	The rotation $R_{\theta}: \R^2 \to \R^2$ of angle $\theta$ about the origin is linear.
	\\
	\vspace{0.3cm}
	\textbf{Exercise}: what is the canonical matrix of $R_{\theta}$ ?
\end{frame}

\section{Operations on matrices}

\begin{frame}[t]{Addition and scalar multiplication}
	Sum of two matrices of the \textbf{same} dimensions:
	{\small
		$$
		\!\!\!\!\!\!\!\!\!\!\!\!
		\begin{pmatrix}
			a_{1,1}  & \cdots & a_{1,m} \\
			\vdots & \ddots & \vdots \\
			a_{n,1} & \cdots & a_{n,m} \\
		\end{pmatrix}
		+
		\begin{pmatrix}
			b_{1,1}  & \cdots & b_{1,m} \\
			\vdots & \ddots & \vdots \\
			b_{n,1} & \cdots & b_{n,m} \\
		\end{pmatrix}
		=
		\begin{pmatrix}
			a_{1,1} + b_{1,1}  & \cdots & a_{1,m} + b_{1,m} \\
			\vdots & \ddots & \vdots \\
			a_{n,1} + b_{n,1} & \cdots & a_{n,m} + b_{n,m} \\
		\end{pmatrix}
		$$
	}

	\vspace{2cm}
	{
		Multiplication by a scalar $\lambda$:
		$$
		\lambda
		\begin{pmatrix}
			a_{1,1}  & \cdots & a_{1,m} \\
			\vdots & \ddots & \vdots \\
			a_{n,1} & \cdots & a_{n,m} \\
		\end{pmatrix}
		=
		\begin{pmatrix}
			\lambda a_{1,1}  & \cdots & \lambda a_{1,m} \\
			\vdots & \ddots & \vdots \\
			\lambda a_{n,1} & \cdots & \lambda a_{n,m} \\
		\end{pmatrix}
		$$
	}
\end{frame}
\begin{frame}[t]{Product of two matrices}
	\textbf{Warning:}
	{\small
		$$
		\!\!\!\!\!\!\!\!\!\!\!\!
		\begin{pmatrix}
			a_{1,1}  & \cdots & a_{1,m} \\
			\vdots & \ddots & \vdots \\
			a_{n,1} & \cdots & a_{n,m} \\
		\end{pmatrix}
		\times
		\begin{pmatrix}
			b_{1,1}  & \cdots & b_{1,m} \\
			\vdots & \ddots & \vdots \\
			b_{n,1} & \cdots & b_{n,m} \\
		\end{pmatrix}
		\neq
		\begin{pmatrix}
			a_{1,1} \times b_{1,1}  & \cdots & a_{1,m} \times b_{1,m} \\
			\vdots & \ddots & \vdots \\
			a_{n,1} \times b_{n,1} & \cdots & a_{n,m} \times b_{n,m} \\
		\end{pmatrix}
		$$
	}
\end{frame}

\begin{frame}[t]{Matrix product}
	Let $L \in \R^{n \times m}$ and $M \in \R^{m \times k}$. 
	%\vspace{2cm}
	\begin{definition}[Matrix product]
		The matrix product $LM$ is the $n \times k$ matrix of the linear map $L \circ M$.
	\end{definition}
\end{frame}
\begin{frame}[t]{Rotations in $\R^2$}
	The $R_{a}$ and $R_{b}$ denote respectively the matrix of the rotation of angle $a$ and $b$ about the origin, in $\R^2$.
	\\
	\vspace{0.3cm}
	\textbf{Exercise}:
	Compute the product $R_a R_b$.

\end{frame}
\begin{frame}[t]{Matrix product properties}
\end{frame}

\section{Kernel and image}

\begin{frame}[t]{Definitions}
	Let \quad $L: \R^m \to \R^n$ \quad be a linear transformation .
	\begin{definition}[Kernel]
		The kernel $\Ker(L)$ (or nullspace) of $L$ is defined as the set of all vectors $v \in \R^m$ such that $L(v) = 0$, i.e.
		$$
		\Ker(L) \defeq \big\{ v \in \R^m \, \big| \, L(v) = 0 \big\}.
		$$
	\end{definition}

	\begin{definition}[Image]
		The image $\Im(L)$ (or column space) of $L$ is defined as the set of all vectors $u \in \R^n$ such that there exists $v \in \R^m$ such that $L(v) = u$. 
	\end{definition}
	\textbf{Remark:} $\Im(L)$ is also the Span of the columns of the matrix representation of $L$.
\end{frame}



\begin{frame}{Picture}
\end{frame}


\begin{frame}[t]{Example: orthogonal projection}
	Consider \quad $L: \R^2 \to \R^2$ \quad to be the orthogonal projection onto the $x$-axis.
\end{frame}


\section{Why do we care about this ?}

\begin{frame}[t]{Linear systems}


\end{frame}

\appendix
\backupbegin
\begin{frame}
	\frametitle{Questions?}
\end{frame}
\backupend

\end{document}
