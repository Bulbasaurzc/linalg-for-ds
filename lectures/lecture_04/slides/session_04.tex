\documentclass{beamer}

\usepackage{../../../latex_style/beamerthemeExecushares}
\usepackage{../../../latex_style/notations}

\title{Session 4: Norms and inner-products}
\subtitle{Optimization and Computational Linear Algebra for Data Science}
\author{Léo Miolane}
\date{}

\setcounter{showSlideNumbers}{1}

\begin{document}
\setcounter{showProgressBar}{0}
\setcounter{showSlideNumbers}{0}

\frame{\titlepage}

\begin{frame}
	\frametitle{Contents}
	\begin{enumerate}
		\item Why do we care about all these things ? \below{Application to data science: image compression}
		\item Norms \& inner-products
		\item Orthogonality
		\item Orthogonal projection
		\item Proof of the Cauchy-Schwarz inequality
	\end{enumerate}
\end{frame}


\setcounter{framenumber}{0}
\setcounter{showSlideNumbers}{1}

\section{Orthogonality}

\begin{frame}{Definition}
\begin{definition}[Orthogonality]
	\begin{itemize}
		\item We say that vectors $x$ and $y$ are \emph{orthogonal} if $\langle x,y \rangle = 0$. We write then $x \perp y$.
		\item We say that a vector $x$ is orthogonal to a set of vectors $A \subset V$ if $x$ is orthogonal to all the vectors in $A$, i.e.\ $\forall y \in A, \ \langle x,y\rangle = 0$. We write then $x \perp A$.
		\item More generality we say that $A \subset V$ and $B \subset V$ are orthogonal if $\langle x,y \rangle = 0$ for all $x \in A$ and all $y \in B$. As before, we write $A \perp B$.
	\end{itemize}
\end{definition}


\begin{exercise}
	If $x$ is orthogonal to $v_1, \dots, v_k$ then $x$ is orthogonal to any linear combination of these vectors i.e.\ $x \perp \Span(v_1, \dots, v_k)$.
\end{exercise}
\end{frame}

\begin{frame}{Pythagorean Theorem}
	\grid

\begin{theorem}[Pythagorean theorem]
	Let $x,y \in V$. Then
	$$
	x \perp y \ \Longleftrightarrow \|x+y\|^2 = \|x\|^2 + \|y\|^2.
	$$
\end{theorem}
\begin{proof}
	\vfill
	\vspace{4cm}
\end{proof}
\end{frame}

\begin{frame}[t]{Orthogonal \& orthonormal families}
	\grid

\begin{definition}
	Let $v_1, \dots, v_k$ be vectors of $V$. We say that the family of vectors $(v_1, \dots, v_k)$ is
	\begin{itemize}
		\item \emph{orthogonal} if the vectors $v_1, \dots, v_n$ are pairwise orthogonal, i.e.\ $\langle v_i, v_j \rangle = 0$ for all $i \neq j$.
		\item \emph{orthonormal} if it is orthogonal and if all the $v_i$ have unit norm: $\|v_1\| = \dots = \|v_k\| = 1$.
	\end{itemize}
\end{definition}
\end{frame}

\begin{frame}[t]{A toy example}
Orthonormal basis are particularly convenient for computing coordinates of vectors:

\begin{block}{Proposition}
	Assume that $\dim(V)=n$ and let $(v_1, \dots, v_n)$ be an \textbf{orthonormal} basis of $V$. Then the coordinates of a vector $x \in V$ in the basis $(v_1, \dots, v_n)$ are $(\langle v_1, x\rangle, \dots, \langle v_n,x \rangle)$:
	$$
	x = \langle v_1, x \rangle v_1 + \cdots + \langle v_n, x \rangle v_n.
	$$
	Moreover, for all $y \in V$, we have $\langle x,y \rangle = \langle v_1,x\rangle\langle v_1,y\rangle + \cdots + \langle v_n,x\rangle\langle v_n,y\rangle$. Taking $y=x$ leads to
	$$
	\|x\| = \sqrt{\langle v_1, x \rangle^2 + \cdots + \langle v_n, x \rangle^2}.
	$$
\end{block}

\end{frame}

\appendix
\backupbegin
\begin{frame}
	\frametitle{Questions?}
\end{frame}
\backupend

\end{document}
