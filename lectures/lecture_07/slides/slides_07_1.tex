\documentclass{beamer}

\usepackage{../../../latex_style/beamerthemeExecushares}
\usepackage{../../../latex_style/notations}

\title{Lecture 7.1: Consequences of the spectral theorem}
\subtitle{Optimization and Computational Linear Algebra for Data Science}
\author{Léo Miolane}
\date{}

\setcounter{showSlideNumbers}{1}

\begin{document}
\setcounter{showProgressBar}{0}
\setcounter{showSlideNumbers}{0}

\frame{\titlepage}
\setcounter{framenumber}{0}
%\setcounter{showProgressBar}{1}
\setcounter{showSlideNumbers}{1}


\begin{frame}[t]{The Spectral theorem}
	\begin{block}{Theorem}
	Let $A \in \R^{n \times n}$ be a \textbf{symmetric} matrix. Then there is a orthonormal basis of $\R^n$ composed of eigenvectors of $A$.
	\end{block}

	That means that if $A$ is symmetric, then there exists an orthonormal basis $(v_1, \dots, v_n)$ of $\R^n$ and $\lambda_1,\dots, \lambda_n \in \R$ such that
	$$
	Av_i = \lambda_i v_i \qquad \text{for all } \ \ i \in \{1, \dots, n\}.
	$$

	\begin{block}{Theorem (Matrix formulation)}
	Let $A \in \R^{n \times n}$ be a \textbf{symmetric} matrix. Then there exists an orthogonal matrix $P$ and a diagonal matrix $D$ of sizes $n \times n$ such that
	$$
		A = P D P^{\sT}.
	$$
	\end{block}
\end{frame}


\begin{frame}[t]{Consequences}
	If 
	$$
	A = P
	\begin{pmatrix}
		\lambda_1 & 0 & \cdots & 0 \\
		0 & \lambda_2 & & \vdots \\
		\vdots &  & \ddots & 0 \\
		0 & \cdots & 0 & \lambda_n \\
	\end{pmatrix}
	P^{\sT}
	$$
	for some orthogonal matrix $P$ then:

	\vspace{1cm}

\textbf{Consequence \#1}: $\lambda_1, \dots, \lambda_n$ are the only eigenvalues of $A$, and the number of time that an eigenvalue appear on the diagonal equals its multiplicity.
\end{frame}

\begin{frame}[t]{Proof sketch on an example}
	\grid

	Consider $n=3$ and 
	$$
	A = P
	\begin{pmatrix}
		3 & 0 &  0 \\
		0 & 3 & 0 \\
		0 & 0 & -1 
	\end{pmatrix}
	P^{\sT}
	\qquad
	\text{where}
	\qquad
	P=
	\begin{pmatrix}
		| & | &  | \\
		v_1 & v_2 & v_3 \\
		| & | & |
	\end{pmatrix}
	$$
	is an orthogonal matrix.
\\

\end{frame}
\begin{frame}[t]{Consequences}
	If 
	$$
	A = P
	\begin{pmatrix}
		\lambda_1 & 0 & \cdots & 0 \\
		0 & \lambda_2 & & \vdots \\
		\vdots &  & \ddots & 0 \\
		0 & \cdots & 0 & \lambda_n \\
	\end{pmatrix}
	P^{\sT}
	$$
	for some orthogonal matrix $P$ then:

	\vspace{1cm}

\textbf{Consequence \#2}: The rank of $A$ equals to the number of non-zero $\lambda_i$'s on the diagonal:
$$
\rank(A) = \# \big\{ i \, \big| \, \lambda_i \neq 0 \big\}.
$$
\end{frame}
\begin{frame}[t]{Proof}
	\grid

\end{frame}
\begin{frame}[t]{Consequences}
	If 
	$$
	A = P
	\begin{pmatrix}
		\lambda_1 & 0 & \cdots & 0 \\
		0 & \lambda_2 & & \vdots \\
		\vdots &  & \ddots & 0 \\
		0 & \cdots & 0 & \lambda_n \\
	\end{pmatrix}
	P^{\sT}
	$$
	for some orthogonal matrix $P$ then:

	\vspace{1cm}

	\textbf{Consequence \#3}: $A$ is invertible if and only if $\lambda_i \neq 0$ for all $i$. In such case
	$$
	A^{-1} = P
	\begin{pmatrix}
		1/\lambda_1 & 0 & \cdots & 0 \\
		0 & 1/\lambda_2 & & \vdots \\
		\vdots &  & \ddots & 0 \\
		0 & \cdots & 0 & 1/\lambda_n \\
	\end{pmatrix}
	P^{\sT}
	$$
\end{frame}

\begin{frame}[t]{Proof}
	\grid

\end{frame}

\begin{frame}[t]{Consequences}
	\grid

	If 
	$$
	A = P
	\begin{pmatrix}
		\lambda_1 & 0 & \cdots & 0 \\
		0 & \lambda_2 & & \vdots \\
		\vdots &  & \ddots & 0 \\
		0 & \cdots & 0 & \lambda_n \\
	\end{pmatrix}
	P^{\sT}
	$$
	for some orthogonal matrix $P$ then:

	\vspace{1cm}

	\textbf{Consequence \#4}: \ \ $\Tr(A) = \lambda_1 + \cdots + \lambda_n$.
\end{frame}

\end{document}
