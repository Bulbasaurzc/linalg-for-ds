\documentclass{beamer}

\usepackage{../../../latex_style/beamerthemeExecushares}
\usepackage{../../../latex_style/notations}

\title{Lecture 5.2: Eigenvalues \& Eigenvectors}
\subtitle{Optimization and Computational Linear Algebra for Data Science}
\author{Léo Miolane}
\date{}

\setcounter{showSlideNumbers}{1}

\begin{document}
\setcounter{showProgressBar}{0}
\setcounter{showSlideNumbers}{0}

\frame{\titlepage}
\setcounter{framenumber}{0}
%\setcounter{showProgressBar}{1}
\setcounter{showSlideNumbers}{1}

\begin{frame}[t]{Intro}
	\grid

\end{frame}

\begin{frame}[t]{Definition}
	\grid

	\vspace{-0.4cm}
	\begin{definition}\label{def:eigen}
		Let $A \in \R^{n \times n}$. A \textbf{non-zero} vector $v \in \R^n$ is said to be an \emph{eigenvector} of $A$ is there exists $\lambda \in \R$ such that
		$$
		A v = \lambda v.
		$$
		The scalar $\lambda$ is called the eigenvalue (of $A$) associated to $v$. 
	\end{definition}
\end{frame}

\begin{frame}[t]{Matrix with no eigenvalues/vectors}
	\grid

\end{frame}

\begin{frame}[t]{Eigenspaces}
	\grid

	\vspace{-0.4cm}
	\begin{definition}
		If $\lambda \in \R$ is an eigenvalue of $A \in \R^{n \times n}$, the set
		$$
		E_{\lambda}(A) = \big\{ x \in \R^n \, \big| \, Ax = \lambda x \big\} = \Ker(A-\lambda \Id)
		$$
		is called the eigenspace of $A$ associated to $\lambda$. The dimension of $E_{\lambda}(A)$ is called the multiplicity of the eigenvalue $\lambda$.
	\end{definition}
\end{frame}


\end{document}
