\documentclass{beamer}

\usepackage{../../../latex_style/beamerthemeExecushares}
\usepackage{../../../latex_style/notations}

\title{Lecture 5.2: Orthogonal matrices}
\subtitle{Optimization and Computational Linear Algebra for Data Science}
\author{Léo Miolane}
\date{}

\setcounter{showSlideNumbers}{1}

\begin{document}
\setcounter{showProgressBar}{0}
\setcounter{showSlideNumbers}{0}

\frame{\titlepage}
\setcounter{framenumber}{0}
%\setcounter{showProgressBar}{1}
\setcounter{showSlideNumbers}{1}

\begin{frame}[t]{Orthogonal matrices}
	\grid

	\begin{definition}
		A matrix $A \in \R^{n \times n}$ is called an \emph{orthogonal matrix} if its columns are an orthonormal family.
	\end{definition}
\end{frame}

\begin{frame}[t]{A proposition}
	\grid

	\vspace{-0.4cm}
	\begin{block}{Proposition}
		Let $A \in \R^{n \times n}$. The following points are equivalent:
		\begin{enumerate}
			\item $A$ is orthogonal.
			\item $A^{\sT} A = \Id_n$.
			\item $A A^{\sT} = \Id_n$
		\end{enumerate}
	\end{block}
\end{frame}

\begin{frame}[t]{Orthogonal matrices \& norm}
	\grid

	\vspace{-0.4cm}
	\begin{block}{Proposition}
	Let $A \in \R^{n \times n}$ be an orthogonal matrix. Then $A$ preserves the dot product in the sense that for all $x,y \in \R^n$,
	$$
	\langle Ax, Ay \rangle = \langle x,y\rangle.
	$$
	In particular if we take $x=y$ we see that $A$ preserves the Euclidean norm: $\|Ax\| = \|x\|$.
\end{block}
\end{frame}


\end{document}
