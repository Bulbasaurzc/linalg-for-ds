\documentclass{beamer}

\usepackage{../../../latex_style/beamerthemeExecushares}
\usepackage{../../../latex_style/notations}

\title{Lecture 5.1: Gram-Schmidt algorithm}
\subtitle{Optimization and Computational Linear Algebra for Data Science}
\author{Léo Miolane}
\date{}

\setcounter{showSlideNumbers}{1}

\begin{document}
\setcounter{showProgressBar}{0}
\setcounter{showSlideNumbers}{0}

\frame{\titlepage}
\setcounter{framenumber}{0}
%\setcounter{showProgressBar}{1}
\setcounter{showSlideNumbers}{1}

\begin{frame}[t]{Purpose of the algorithm}
	The Gram-Schmidt process takes as 
	\begin{itemize}
		\item \textbf{Input:} a \emph{linearly independent} family $(x_1, \dots, x_k)$ of $\R^n$.
		\item \textbf{Output:} an \emph{orthonormal basis} $(v_1, \dots v_k)$ of $\Span(x_1, \dots, x_k)$.
	\end{itemize}
	\vspace{4cm}

	\begin{block}{Consequence}
		Every subspace of $\R^n$ admits an orthonormal basis.
	\end{block}

\end{frame}

\begin{frame}[t]{Gram-Schmidt algorithm}
	\grid

	The Gram-Schmidt process constructs $v_1,v_2, \dots, v_k$ in this order, such that for all $i \in \{1, \dots, k\}$:
	\begin{align*}
		\mathcal{H}_i: 
		\begin{cases}
			(v_1, \dots, v_i) \ \text{is an orthonormal family} \\
			\Span(v_1, \dots, v_i) = \Span(x_1, \dots, x_i).
		\end{cases}
	\end{align*}
\end{frame}

\begin{frame}[t]{Iterative construction of the $v_i$'s}
	\grid

	\pause
	\pause
\end{frame}


\end{document}
