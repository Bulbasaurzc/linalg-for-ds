\documentclass[11pt,nocut]{article}

\usepackage{../latex_style/packages}
\usepackage{../latex_style/notations}

\title{\vspace{-2.0cm}%
	Optimization and Computational Linear Algebra for Data Science\\
Final review problems}
%\author{Léo \textsc{Miolane} \ $\cdot$ \ \texttt{leo.miolane@gmail.com}}
\date{}
%\setcounter{section}{*}

\begin{document}
\maketitle
%%\noindent\textbf{Rules:}
\centerline{\pgfornament[width=13cm]{89}}
{\small
	\begin{itemize}
		\item Unless otherwise stated, all answers must be mathematically justified.
		\item Partial answers will be graded. 
		\item You can work in groups but each student must write his/her own solution based on his/her own understanding of the problem. Please list on your submission the students you work with for the homework (his will note affect your grade).
		\item Late submissions will be graded with a penalty of $10\%$ per day late. Weekend days do not count: from Friday to Monday count $1$ day.
		\item Problems with a $(\star)$ are extra credit, they will not (directly) contribute to your score of this homework. However, for every $4$ extra credit questions successfully answered your lowest homework score get replaced by a perfect score.
		\item If you have any questions, feel free to contact myself (\texttt{leo.miolane@gmail.com}) or to stop at the office hours.
	\end{itemize}
}
\vspace{-0.4cm}
\centerline{\pgfornament[width=13cm]{89}}
\vspace{0.5cm}

\begin{center}
	{\Large
		For review exercises on linear algebra, look at last year's final review exercises (available the course's website).
	}
\end{center}

\vspace{1mm}

\begin{problem}
	Let $A \in \R^{n \times m}$. Let $\sigma_1(A)$ be the largest singular value of $A$.
	Show that
	$$
	\sigma_1(A) = \max_{\|x\|=1} \|Ax\|.
	$$
\end{problem}

\vspace{2mm}

\begin{problem}
	Let $A \in \R^{n \times m}$. Show that $A^{\sT} A$  and $A A^{\sT}$ have the same non-zero eigenvalues.
\end{problem}

\vspace{2mm}

\begin{problem}[True or false?]
	For each of the following statement, say if they are true or false and justify your answer.
	\begin{itemize}
		\item For all $A \in \R^{n \times n}$, if $\lambda$ is an eigenvalue of $A$ then $\lambda^2$ is an eigenvalue of $A^2$.
		\item For all $A \in \R^{n \times n}$, if $\sigma$ is a singular value of $A$ then $\sigma^2$ is a singular value of $A^2$.
		\item For all symmetric matrix $A \in \R^{n \times n}$ the eigenvalues of $A$ are singular values of $A$.
	\end{itemize}
\end{problem}

\begin{problem}
	Let $A \in \R^{n \times m}$. Show that for all $u \in \Im(A)$ and for all $v \in \Ker(A^{\sT})$ we have
	$$
	\langle u,v \rangle = 0.
	$$
\end{problem}

%\begin{problem}
	%Recall that the infinity norm of $x \in \R^n$ is defined by $\|x\|_{\infty} = \max(|x_1|, \dots, |x_n|)$. Show that
	%$$
	%\|x\|_{\infty} \leq \|x\|_2 \leq \sqrt{n} \|x\|_{\infty}.
	%$$
%\end{problem}

%\begin{problem}
	%Let $A=U \Sigma V^{\sT}$ denote the singular value decomsition of $A \in \R^{n \times n}$. Using this decomposition, construct the matrix of the orthogonal projection onto $\Im(A)$.
%\end{problem}

\begin{problem}
	Let $A \in \R^{n \times m}$. Show that if $A$ has linearly independent columns, then $A^{\dagger} = (A^{\sT}A)^{-1} A^{\sT}$.
\end{problem}

\begin{problem}
	Which of the following functions $f: \R^n \to \R^n$ are convex? Justify your answer
	\begin{itemize}
		\item $f(x) = \|x\|^2$.
		\item $f(x) = Ax$, for some $A \in \R^{n \times n}$.
		\item $f(x) = \sum_{i=1}^n x_i^3$.
	\end{itemize}
\end{problem}

\begin{problem}
	Which of the following subset $S$ of $\R^n$ are convex? Justify your answer
	\begin{itemize}
		\item $S = \{x \in \R^n \, | \, \|x\|_{\infty} \leq 1\}$.
		\item $S = \{x \in \R^n \, | \, \|x\|_{1} \geq 1\}$.
		\item $S = \{x \in \R^n \, | \, \|Ax\| < 1\}$, for some $A \in \R^{n \times n}$.
	\end{itemize}
\end{problem}


\begin{problem}
	Show that we are performing PCA on $n$ data points $a_1, \dots, a_n \in \R^d$ and keep only the first $k < d$ principal components of each point. We store the dimensionally reduced dataset in a $n \times k$ matrix $B$, where $B_{i,j}$ is the $j^{\rm th}$ principal component of the point $a_i$. Show that the columns of $B$ are orthogonal.
\end{problem}

\newpage
\begin{problem}[True of false?]
	For each of the following statement, say if they are true or false and justify your answer.
	\begin{itemize}
		\item If a continuous function $f:\R \to \R$ has a unique minimizer then $f$ is convex.
		\item If a continuous function $f:\R \to \R$ is such that there exists $x_0$ such that $f$ is decreasing on $(-\infty,x_0]$ and increasing on $[x_0, + \infty)$ then $f$ is convex.
		\item A twice differentiable function $f: \R \to \R$ whose derivative $f'$ is non-decreasing is convex.
	\end{itemize}
\end{problem}


\begin{problem}
	Let $f: \R^n \to \R$ be a convex, differentiable function. Assume that there exist $x,y \in \R^n$ such that $\nabla f(x) = \nabla f(y) = 0$. Show that $\nabla f(\frac{1}{2}(x+y)) = 0$.
\end{problem}


\begin{problem}
	Assume that we are doing linear regression with the least-squares cost
	$$
	f(x) = \|Ax - y\|^2
	$$
	where $A \in \R^{n \times d}$ and $y \in \R^n$. Should you normalize the dataset $A$ (that is, should we divide each column of $A$ by its norm) to get better results (smaller training error or smaller test error on new data points)?
	\\

	Suppose that we now want to use the lasso and minimize
	$$
	f(x) = \|Ax - y\|^2 + \lambda \|x\|_1
	$$
	for some $\lambda > 0$. Is there any reason why you might want to normalize the dataset in that case?
\end{problem}


\begin{problem}
	Compute the critical points of the following function and say if they are local minimizers, local maximizers or saddle points.
	$$
	f(x,y,z) = x^2 + y^2 - z^2
	\quad \text{and} \quad g(x,y) = 3x^2 + y^2 - 6x -4y - 10.
	$$
\end{problem}
\begin{problem}
	Solve the following constrained minimization problem (find all the solutions to these problems).
	\begin{enumerate}
		\item Minimize $x + y + z$ subject to $e^{-x} + e^{-y} + e^{-z} = 1$.
		\item Minimize $x^2 + y^2 + z^2$ subject to $xyz=1$.
	\end{enumerate}
\end{problem}

\begin{problem}
	Assume that we are doing standard gradient descent to minimize the least-square cost
	$$
	f(x) = \|Ax-y\|^2.
	$$
	Assume that the columns of  $A$ are linearly dependent, meaning that $\Ker(A) \neq \{0\}$. At which speed should gradient descent converge to the minimum?
	If now $\Ker(A) = \{0\}$, at which speed should gradient descent converge?
	By speed, we only ask about the dependence in $t$, the number of iterations, of the gap 
	$f(x_t) - \min f$, where $x_t$ is the position of gradient descent after $t$ iterations.
\end{problem}

\begin{problem}
	Let $A \in \R^{n \times d}$. Assume that the columns of $A$ are linearly independent. How many steps of Newton's method do you need to minimize
	$$
	\|Ax - y\|^2 \ ?
	$$
	($y \in \R^n$ is a fixed vector). Justify your answer.
\end{problem}


\begin{problem}
	When running stochastic gradient descent, what are upsides and downsides of having a rapidly decaying learning rate?
\end{problem}

\newpage

\begin{center}
	{\Large
	Hints. Please only look at the hints if you have spent a reasonable time thinking about the problems!
	}
\end{center}
\begin{enumerate}
	\item Use the fact that $\|Ax\|^2 = x^{\sT} A^{\sT} A x$ and then use the SVD decomposition of $A$ to rewrite $A^{\sT} A$.
	\item Use the SVD of $A$.
	\item (a) True (b) False (c) False (eigenvalues can be negative but singular values can not. The singular values of a symmetric matrix are the absolute value of its eigenvalues).
	\item Use the definitions of kernel and image.
	\item Use the SVD decomposition of $A$ to compute $(A^{\sT}A)^{-1}A^{\sT}$ and see that it corresponds to the definition of $\A^{\dagger}$.
	\item Convex, convex, not convex.
	\item Convex, not convex, convex.
	\item Express the columns of $B$ using the left-singular vectors of the matrix $A$ whose rows are the $a_i$.
	\item False. False. True.
	\item Show that $(x+y)/2$ is a global minimizer of $f$.
	\item Normalizing the dataset is useless for ordinary least-squares, but can be useful for Lasso.
	\item Compute gradient and Hessian. 
	\item Use Lagrange multipliers.
	\item If the columns of $A$ are linearly dependent, then $f$ will be $L$-smooth but not strongly convex, hence the speed of gradient descent will be $O(1/t)$.
		If the columns of $A$ are linearly independent then you can show that $f(x)$ is $\mu$-strongly convex and $L$-smooth, for some $\mu,L>0$. Hence the error of gradient descent will be $O(e^{-\rho t})$ after $t$ steps, for some constant $\rho>0$. 
	\item 1 step.
	\item See lecture notes.
\end{enumerate}

\vspace{1cm}
\centerline{\pgfornament[width=7cm]{87}}

%\bibliographystyle{plain}
%\bibliography{./references.bib}
\end{document}
