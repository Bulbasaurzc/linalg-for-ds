\documentclass[11pt,nocut]{article}

\usepackage{../latex_style/packages}
\usepackage{../latex_style/notations}

\title{\vspace{-2.0cm}%
	Optimization and Computational Linear Algebra for Data Science\\
Homework 7: Principal component analysis}
%\author{Léo \textsc{Miolane} \ $\cdot$ \ \texttt{leo.miolane@gmail.com}}
\date{\vspace{-1cm}Due on November 5, 2019}
\setcounter{section}{7}

\begin{document}
\maketitle
%\noindent\textbf{Rules:}
\centerline{\pgfornament[width=13cm]{89}}
{\small
	\begin{itemize}
		\item Unless otherwise stated, all answers must be mathematically justified.
		\item Partial answers will be graded. 
		\item You can work in groups but each student must write his/her own solution based on his/her own understanding of the problem. Please list on your submission the students you work with for the homework (his will note affect your grade).
		\item Late submissions will be graded with a penalty of $10\%$ per day late. Weekend days do not count: from Friday to Monday count $1$ day.
		\item Problems with a $(\star)$ are extra credit, they will not (directly) contribute to your score of this homework. However, for every $4$ extra credit questions successfully answered your lowest homework score get replaced by a perfect score.
		\item If you have any questions, feel free to contact myself (\texttt{leo.miolane@gmail.com}) or to stop at the office hours.
	\end{itemize}
}
\vspace{-0.4cm}
\centerline{\pgfornament[width=13cm]{89}}
\vspace{0.5cm}


%DRAFT

%\color{green}

\begin{problem}[3 points]
	Let $A \in \R^{n \times m}$. The Singular Values Decomposition (SVD) tells us that there exists two orthogonal matrices $U \in \R^{n \times n}$ and $V \in \R^{m \times m}$ and a matrix $\Sigma \in \R^{n \times m}$ such that $\Sigma_{1,1} \geq \Sigma_{2,2}  \geq \cdots \geq 0$ and $\Sigma_{i,j} = 0$ for $i\neq j$
	$$
	A = U \Sigma V^{\sT}.
	$$
	The columns $u_1, \dots, u_n$ of $U$ (respectively the columns $v_1, \dots, v_m$ of $V$) are called the left (resp.\ right) singular vectors of $A$. The non-negative numbers $\sigma_i \defeq \Sigma_{i,i}$ are the singular values of $A$. Moreover we also know that $r \defeq \rank(A) = \# \{i \, | \, \Sigma_{i,i} \neq 0 \}$.
	\begin{enumerate}[label=\normalfont(\textbf{\alph*})]
		\item Let 
			$\widetilde{U} = \!
			{\small \begin{pmatrix}
					| & & | \\
					u_1 & \!\cdots\! & u_r \\
					| & & | 
			\end{pmatrix} } \! \in \! \R^{n \times r}$ ,
			$\widetilde{V} = \!
			{\small \begin{pmatrix}
					| & & | \\
					v_1 & \!\cdots\! & v_r \\
					| & & | 
			\end{pmatrix} } \! \in \! \R^{m \times r}$ and
			$\widetilde{\Sigma} = \Diag(\sigma_1, \dots, \sigma_r) \in \R^{r \times r}$.
			Show that $A = \widetilde{U}\widetilde{\Sigma}\widetilde{V}^{\sT}$.
		\item Give orthonormal bases of $\Ker(A)$ and $\Im(A)$ in terms of the singular vectors $u_1, \dots, u_n, v_1, \dots , v_m$.
	\end{enumerate}
\end{problem}

\vspace{1mm}

\begin{problem}[3 points]
	We say that a symmetric matrix $M \in \R^{n \times n}$ is positive definite if for all \textbf{non-zero} $x \in \R^n$,
	$$
	x^{\sT} M x > 0.
	$$
	If a matrix $M$ is positive definite, then $M$ is also positive semi-definite, but the converse is not true. One of the goals of this problem is to prove one of the implications of Proposition~1.2 of the notes (Lecture~7). You are of course not allowed to use this proposition to solve this problem.
	\begin{enumerate}[label=\normalfont(\textbf{\alph*})]
		\item Let $M \in \R^{n \times n}$ be a positive definite matrix. Show that its eigenvalues are all strictly positive and that $M$ is invertible. 
		\item Let $M \in \R^{n \times n}$ be a symmetric matrix. Show that there exists $\alpha > 0$ such that the matrix $M + \alpha \Id_n$ is positive definite.
	\end{enumerate}
\end{problem}

\vspace{1mm}


\begin{problem}[4 points]
	You have been given a mysterious dataset that may contain important informations! This dataset is a collection of $n=3000$ points of dimension $d=1000$.
	Investigate the structure of this dataset using PCA/plots... , and find out if the dataset contains any information.
	\\

	The \texttt{zip} file \texttt{mysterious\_data.zip} contains a text file containing the $3000 \times 1000$ data matrix.
	The \emph{Jupyter notebook} \texttt{mysterious\_data.ipynb} contains a function to read the text file.
	The numpy function \texttt{numpy.linalg.eigh} is great to compute eigenvalues and eigenvectors of a symmetric matrix. 
	\\

	\textbf{It is intended that you code in Python and use the provided Jupyter Notebook. Please only submit a pdf version of your notebook (right-click $\to$ `print' $\to$ `Save as pdf').}
\end{problem}


\vspace{1mm}

%\begin{problem}[3 points]
%\end{problem}

%\vspace{1mm}


\begin{problem}[$\star$]
	Let $M \in \R^{n \times n}$ be a symmetric matrix. Let $\lambda_1 \geq \dots \geq \lambda_n$ be the eigenvalues of $M$. Show that for all $d \leq n$:
	$$
	\max_{\substack{U \in \R^{n\times d} \\ U^{\sT} U = \Id_d}} \Tr(U^{\sT} M U) = \sum_{i=1}^d \lambda_i.
	$$
\end{problem}
\vspace{1cm}
\centerline{\pgfornament[width=7cm]{87}}

%\bibliographystyle{plain}
%\bibliography{./references.bib}
\end{document}
