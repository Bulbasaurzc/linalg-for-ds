\documentclass[11pt,nocut]{article}

\usepackage{../latex_style/packages}
\usepackage{../latex_style/notations}
%\externaldocument{../lecture_02/lecture_02}
\externaldocument{../lecture_07/lecture_07}


\title{\vspace{-2.0cm}%
	Optimization and Computational Linear Algebra for Data Science\\
Lecture 11: Linear regression, matrix completion}
\author{Léo \textsc{Miolane} \ $\cdot$ \ \texttt{leo.miolane@gmail.com}}
\date{\today}

\begin{document}
\maketitle
\textbf{Warning:}
\emph{This material is not meant to be lecture notes. It only gathers the main concepts and results from the lecture, without any additional explanation, motivation, examples, figures...
}


\section{Least squares}

Assume that we are given point $a_i = (a_{i,1}, \dots, a_{i,d}) \in \R^d$ with labels $y_i \in \R$ for $i=1 \dots n$.
We aim at finding a vector $x\in\R^d$ such that
$$
y_i \simeq \langle a_i, x \rangle = \sum_{j=1}^d a_{i,j} x_j, \qquad \text{for} \ \ i=1 \dots n.
$$
If we denote by $A$ the $n \times d$ matrix whose rows are $a_1, \dots, a_n$, i.e.\ $A_{i,j} = a_{i,j}$, we are looking for some $x$ such that $Ax \simeq y$.

\subsection{Solving the system $Ax=y$}

As we have seen in Lecture 2, we can distinguish two cases:
\begin{itemize}
	\item If $y \not\in \Im(A)$ then the equation $Ax=y$ does not admit any solution (by definition of $\Im(A)$).
	\item If $y \in \Im(A)$ then the equation $Ax=y$ admits at least a solution $x_0$ (by definition of $\Im(A)$). Moreover, the set of (all) solutions is
		$$
		x_0 + \Ker(A) = \{x_0 + v \, | \, v \in \Ker(A) \}.
		$$
		In particular, if $\Ker(A) = \{0\}$ then the equation admits a unique solution.
\end{itemize}

In the second case, one can obtain an expression for a particular solution $x_0$ using the SVD of $A$.
Let $r = \rank(A)$, $\sigma_1, \sigma_2, \dots, \sigma_r >0$ be the non-zero singular values of $A$ and $\Sigma = \Diag(\sigma_1, \dots, \sigma_r)$. Finally, let $A = U \Sigma V^{\sT}$ be the SVD of $A$, where $V \in \R^{n \times r}$ and $U \in \R^{d \times r}$ are matrices that have orthonormal columns. 
\\

Notice that $V^{\sT} V = \Id$ and that $U U^{\sT}$ is the orthogonal projection on $\Im(A)$. Hence, if we let $x_0 = V \Sigma^{-1} U^{\sT} y$, we have
$$
A x_0 =U \Sigma V^{\sT}V \Sigma^{-1} U^{\sT} y
= U U^{\sT} y = y
$$
because we assumed that $y \in \Im(A)$.
This motivates the following definition:
\begin{definition}[Moore-Penrose pseudo-inverse]
	The matrix $A^{\dagger} \defeq V \Sigma^{-1} U^{\sT}$ is called the (Moore-Penrose) pseudo-inverse of $A$.
\end{definition}

Notice that in the case where $A$ is invertible, $A^{\dagger} = A^{-1}$.
From the analysis above, we deduce:
\begin{proposition}\label{prop:linear_system}
	The set of solution of the linear system $Ax = y$ is
	\begin{itemize}
		\item $\emptyset$ if $y \not\in \Im(A)$.
		\item $A^{\dagger}y + \Ker(A)$ otherwise.
	\end{itemize}
\end{proposition}

\subsection{Least squares}

In general, there is no reason for $y$ to belong to $\Im(A)$, especially when $n > d$. (Exercise: why?)
Therefore one is rather interested by solving
\begin{equation}\label{eq:least_squares}
\min_{x \in \R^d} \| Ax - y \|^2.
\end{equation}
The function $f: x \mapsto \|Ax - y\|^2$ is convex (Exercise: why?) and differentiable. Hence
$x$ is solution of \eqref{eq:least_squares} if and only if $\nabla f (x) = 0$. Compute
$$
f(x) = (Ax - y)^{\sT}(Ax - y) = x^{\sT} A^{\sT} A x - 2 y^{\sT} A x + \|y\|^2.
$$
Hence $\nabla f(x) = 2 A^{\sT} A x - 2 A^{\sT} y$. We conclude
$$
x \ \ \text{is solution of} \ \ \eqref{eq:least_squares} \qquad
\Longleftrightarrow
\qquad A^{\sT} A x = A^{\sT} y.
$$
If $A^{\sT} A$ is invertible there is a unique minimizer $x^* = (A^{\sT} A)^{-1} A^{\sT} y$.
In the general case, we see that the solutions of \eqref{eq:least_squares} are the solutions of the linear system $A^{\sT} A x = A^{\sT} y$. From Proposition \ref{prop:linear_system} we get that the solutions of \eqref{eq:least_squares} are
$$
(A^{\sT}A)^{\dagger} A^{\sT}y + \Ker(A^{\sT} A).
$$
This expression simplifies a lot. First (exercise!) we have $\Ker(A^{\sT} A) = \Ker(A)$.
Then if we let $A = U \Sigma V^{\sT}$ be the SVD of $A$, we have
$$
A^{\sT} A = V \Sigma^2 V^{\sT}.
$$
$V \Sigma^2 V^{\sT}$ is therefore the SVD of $A^{\sT} A$. Hence $(A^{\sT} A)^{\dagger} = V \Sigma^{-2} V^{\sT}$.
This gives $(A^{\sT}A)^{\dagger} A^{\sT} = V \Sigma^{-2} V^{\sT}V \Sigma U^{\sT} = A^{\dagger}$. We conclude:

\begin{proposition}[Least squares]\label{prop:least_squares}
	The set of solution of the minimization problem $\min_{x \in \R^n} \|Ax - y\|^2$ is
	$$
	A^{\dagger} y + \Ker(A).
	$$
\end{proposition}

\section{Penalized least squares: Ridge regression and Lasso}

When $\Ker(A) \neq \emptyset$ the least squares problem \eqref{eq:least_squares} has an infinite number of solutions: which one should we pick?

\subsection{Ridge regression}

The Ridge regression adds a $\ell_2$ penalty to the least square problem in order to ``select'' a solution of smaller norm, and minimizes
\begin{equation}\label{eq:ridge}
	\min_{x \in \R^d} \Big\{ \| Ax - y \|^2 + \lambda \|x\|^2 \Big\},
\end{equation}
for some penalization parameter $\lambda >0$.
\begin{exercise}
	Show that \eqref{eq:ridge} admits a unique solution given by
	$$
	x^{\rm Ridge} = (A^{\sT} A + \lambda \Id)^{-1} A^{\sT} y.
	$$
\end{exercise}

\subsection{Lasso}

The Lasso adds a $\ell_1$ penalty to the least square problem, and minimizes
\begin{equation}\label{eq:lasso}
	\min_{x \in \R^d} \Big\{ \frac{1}{2} \| Ax - y \|^2 + \lambda \|x\|_1 \Big\},
\end{equation}
for some penalization parameter $\lambda >0$.
The Lasso has the wonderful property of \emph{feature selection}: the solution $x^*$ of \eqref{eq:lasso} is likely to be \emph{sparse} (many coordinates $x^*_j$ will be set to $0$).
In words, the Lasso estimator discards the ``useless features'' by setting its corresponding coefficient to $0$. 
This is particularly nice for the interpretability of the results: in many application, each data point $a_i$ has an enormous number of features ($d$ very large), but only a small number of them are useful to predict the label $y_i$.
\\

We can gain some intuition on this phenomena on Figure.

\\

\paragraph{Lasso for orthonormal design}
In general there is no closed form formula for the Lasso estimator. It is however possible to derive one in the case where the matrix $A$ has orthonormal columns: $A^{\sT} A = \Id$. 
The least square estimator is then given by $x^{\rm LS} = A^{\sT} y$ and the Lasso cost function becomes
\begin{equation}\label{eq:lasso_o}
\frac{1}{2} \| Ax - y \|^2 + \lambda \|x\|_1 
= \frac{1}{2} \|x\|^2 - \langle x^{\rm LS}, x \rangle + \frac{1}{2} \|y\|^2 + \lambda \|x\|_1.
\end{equation}
The minimizer of \eqref{eq:lasso_o} is therefore the minimizer of
$$
\frac{1}{2} \|x\|^2 - \langle x^{\rm LS}, x \rangle + \lambda \|x\|_1
=
\sum_{j=1}^d \frac{x_j^2}{2} - x^{\rm LS}_j x_j + \lambda |x_j| 
=
\sum_{j=1}^d f_{x_j^{\rm LS}}(x_j),
$$
where $f_{x_0}(x) \defeq \frac{1}{2}x^2 - x_0 x + \lambda |x|$.

\begin{exercise}
	Show that the function $f_{x_0}$ admits a unique minimizer given by
	$$
	x^* = \eta(x_0; \lambda),
	$$
	where $\eta$ denotes the ``soft-thresholding'' function:
	$$
	\eta(x_0;\lambda) = 
	\begin{cases}
		x_0-\lambda & \text{if} \quad x_0 \geq \lambda \\
		0 & \text{if} \quad -\lambda \leq x_0 \leq \lambda \\
		x_0 + \lambda & \text{if} \quad x_0 \leq -\lambda.
	\end{cases}
	$$
\end{exercise}
\vspace{2mm}

We conclude that the Lasso estimator is given by
$$
x^{\rm Lasso}_{j} = \eta\big(x_j^{\rm LS}; \lambda\big) \qquad \text{for} \quad j=1, \dots, d.
$$
This formula confirms the intuition of Figure: the Lasso estimator translates the coefficients of the least-square solution by $\lambda$, truncating at $0$.

\section{Norms for matrices}

Before looking at low-rank matrix estimation and matrix completion, we need to look at different norms we can have for matrices. Recall that $\R^{n \times m}$, the set of $n \times m$ matrices, is a vector space (of dimension $nm$).
\\

\paragraph{The Frobenius norm.} The most obvious norm to consider is the equivalent of the $\ell_2$ norm, called the Frobenius norm:
\begin{definition}[Frobenius norm]
	The Frobenius norm of a matrix $A \in \R^{n \times m}$ is defined as
	$$
	\|A\|_F = \sqrt{\sum_{i=1}^n \sum_{j=1}^m A_{i,j}^2}. 
	$$
\end{definition}

\begin{remark}
	The Frobenius norm is the norm induced by the following inner-product on matrices:
	$$
	\langle A,B \rangle = \Tr(A^{\sT}B) = \sum_{i=1}^n\sum_{j=1}^m A_{i,j} B_{i,j}, \qquad \text{for} \ \ A,B \in \R^{n \times m}.
	$$
\end{remark}

\begin{proposition}
	Let $A \in \R^{n \times n}$ and let $\sigma_1, \dots, \sigma_{\min(n,m)}$ be the singular values of $A$. Then
	$$
	\|A\|_F = \sqrt{\sum_{i=1}^{\min(n,m)} \sigma_i^2}.
	$$
\end{proposition}

\paragraph{The spectral norm.} Another extremely common norm for matrices is the spectral norm, also called the operator norm:

\begin{definition}[Spectral norm]
	The spectral norm of a matrix $A \in \R^{n \times m}$ is defined as the maximal singular value of $A$:
	$$
	\|A\|_{\rm Sp} = \max_{\substack{x \in \R^m \\ \|x\|=1}} \|Ax\|.
	$$
\end{definition}

\paragraph{The nuclear norm.} We have seen above that the Frobenius norm of a matrix correspond to the $\ell_2$ norm of its singular values and that the spectral norm correspond to the infinity norm of the singular values. The nuclear norm is simply the $\ell_1$ norm of its singular values:
\begin{definition}[Nuclear norm]
	Let $A \in \R^{n \times n}$ and let $\sigma_1, \dots, \sigma_{\min(n,m)}$ be the singular values of $A$. The nuclear norm of $A$ is defined as:
	$$
	\|A\|_* = \sum_{i=1}^{\min(n,m)} \sigma_i.
	$$
	
\end{definition}

\section{Low-rank matrix estimation and matrix completion}



\section*{Further reading}


\vspace{1cm}
\centerline{\pgfornament[width=7cm]{71}}


\bibliographystyle{plain}
\bibliography{../references.bib}
\end{document}
