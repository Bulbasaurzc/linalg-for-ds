\documentclass{beamer}

\usepackage{../../latex_style/beamerthemeExecushares}
\usepackage{../../latex_style/notations}

\title{Lecture 3.2: Some properties of the rank}
\subtitle{Optimization and Computational Linear Algebra for Data Science}
\author{Léo Miolane}
\date{}

\setcounter{showSlideNumbers}{1}

\begin{document}
\setcounter{showProgressBar}{0}
\setcounter{showSlideNumbers}{0}

\frame{\titlepage}



\setcounter{framenumber}{0}
%\setcounter{showProgressBar}{1}
\setcounter{showSlideNumbers}{1}

\begin{frame}[t]{The rank-nullity theorem}
	\begin{theorem}
		Let $L: \R^m \to \R^n$ be a linear transformation. Then
		$$
		\rank(L) + \dim(\Ker(L)) = m.
		$$
	\end{theorem}
\end{frame}

\begin{frame}[t]{Some inequalities}
	\vspace{-0.5cm}
	\begin{block}{Proposition}
		Let $A \in \R^{n \times m}$ and $B \in \R^{m \times k}$. Then the following holds
		\begin{enumerate}
			\item $\rank(A) \leq \min(n,m)$.
			\item $\rank(AB) \leq \min(\rank(A),\rank(B))$.
		\end{enumerate}
	\end{block}
	\vspace{-0.1cm}
	\begin{proof}
		\vfill
		\vspace{4.1cm}
	\end{proof}
	\pause
	\pause
\end{frame}



\end{document}
