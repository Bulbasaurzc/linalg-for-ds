\documentclass[11pt,nocut]{article}

%\makeatletter
      %\newwrite\tf@toc
      %\immediate\openout\tf@toc\jobname.toc\relax
%\makeatother

\usepackage{../latex_style/packages}
\usepackage{../latex_style/notations}
\externaldocument{../lecture_02/lecture_02}


\title{\vspace{-2.0cm}%
	Optimization and Computational Linear Algebra for Data Science\\
	Lecture 3: Rank}
	\author{Léo \textsc{Miolane} \ $\cdot$ \ \texttt{leo.miolane@gmail.com}}
\date{\today}

\begin{document}
\maketitle
\textbf{Warning:}
\emph{This material is not meant to be lecture notes. It only gathers the main concepts and results from the lecture, without any additional explanation, motivation, examples, figures...
}


\section{More on basis}

\begin{proposition}\label{prop:complete}
	Let $V$ be a vector space of dimension $n$.
	Let $x_1, \dots, x_k \in V$. 
	If $x_1, \dots, x_k$  are linearly independent then one can find vectors $x_{k+1}, \dots, x_{n} \in V$ such that $(x_1, \dots, x_n)$ forms a basis of $V$.
\end{proposition}

%\begin{proposition}\label{prop:extract}
	%Let $x_1, \dots, x_k \in V$ and write $r = \dim(\Span(x_1, \dots, x_k))$. Then there exists $i_1, \dots i_r \in \{1 \dots k\}$ such that $(x_{i_1}, \dots, x_{i_r})$ forms a basis of $\Span(x_1, \dots, x_k)$.
%\end{proposition}

\section{Definition of the rank}

\begin{definition}[Rank of a family of vectors]
	We define the rank of a family $x_1, \dots, x_k$ of vectors of $\R^n$ as the dimension of its span:
	$$
	\rank(x_1, \dots, x_k) \defeq \dim (\Span(x_1, \dots, x_k)).
	$$
\end{definition}

If the vectors $x_1, \dots x_k$ are linearly independent then $\rank(x_1, \dots x_k) = k$. Indeed, in that case $(x_1, \dots, x_k)$ forms a basis of $\Span(x_1, \dots, x_k)$ so $\dim(\Span(x_1, \dots, x_k)) = k$.


\begin{definition}[Rank of a matrix]\label{def:rank}
	Let $M \in \R^{n \times m}$. Let $\ell_1, \dots, \ell_n \in \R^m$ be the lines of $M$ and $c_1, \dots, c_m \in \R^n$ be its columns.
	Then we have
	\begin{equation}\label{eq:rank}
	\rank(\ell_1, \dots, \ell_n) = \rank(c_1, \dots, c_m).
	\end{equation}
	The rank of the matrix $M$ is then defined as $\rank(M) \defeq\rank(\ell_1, \dots, \ell_n) = \rank(c_1, \dots, c_m)$.
\end{definition}

Since $\Im(M) = \Span(c_1, \dots, c_m)$ an equivalent definition is $\rank(M) = \dim(\Im(M))$.
\\

\begin{proof}
	In order to prove \eqref{eq:rank} it suffices to show (since columns and rows are playing exchangeable roles) that $\rank(\ell_1, \dots, \ell_n) \leq \rank(c_1, \dots, c_m)$.
	Let $r \defeq \rank(\ell_1, \dots, \ell_n)$ and $(x_1, \dots, x_r)$ be a basis of $\Span(\ell_1, \dots, \ell_n)$. We will prove that
	\begin{equation}\label{eq:goal_proof}
	(M x_1, \dots, M x_r) \ \text{is linearly independent}.
	\end{equation}
	The result follows. Indeed $(M x_1, \dots, M x_r)$ is then a linearly independent family of $r$ vectors of $\Im(M) = \Span(c_1, \dots, c_m)$: this implies that $\rank(c_1, \dots, c_m) = \dim(\Span(c_1, \dots, c_m)) \geq r = \rank(\ell_1, \dots, \ell_n)$.

	It remains to prove \eqref{eq:goal_proof}. Let $\alpha_1, \dots, \alpha_r \in \R$ such that $\alpha_1 M x_1 + \cdots + \alpha_r M x_r = 0$. We will show that in such case the $\alpha_i$ are all zero. Define $v \defeq \alpha_1 x_1 + \cdots + \alpha_r x_r$. We have by linearity
	$$
	M v = M (\alpha_1 x_1 + \cdots + \alpha_r x_r) = \alpha_1 M x_1 + \cdots + \alpha_r M x_r = 0.
	$$
	Since the $i^{\rm th}$ coordinate of $Mv$ is equal to $(Mv)_i = \ell_i \cdot v$, we get that $v$ is orthogonal to all the $\ell_i$, and therefore to $\Span(\ell_1, \dots, \ell_n)$. Notice now that $v \in \Span(x_1, \dots, x_r) = \Span(\ell_1, \dots, \ell_n)$ by construction. The vector $v$ is orthogonal to itself, hence $\alpha_1 x_1 + \cdots + \alpha_r x_r = v= 0$. Recall that the family $(x_1, \dots, x_r)$ is linearly independent (because it is a basis) so $\alpha_1 = \cdots = \alpha_r = 0$.
\end{proof}

\section{Properties of the rank}

\begin{proposition}
	Let $A \in \R^{n \times m}$ and $B \in \R^{m \times k}$. Then the following holds
	\begin{enumerate}[label=(\roman*)]
		\item $\rank(A) \leq \min(n,m)$.
		\item $\rank(AB) \leq \min(\rank(A),\rank(B))$.
	\end{enumerate}
\end{proposition}

\begin{theorem}
	Let $M \in \R^{n \times m}$. The following points are equivalent:
	\begin{enumerate}[label=(\roman*)]
		\item \label{item:th_i} $M$ is invertible.
		\item \label{item:th_ii} $\rank(M) = n$.
		\item \label{item:th_iii} $\Ker(M) = \{0\}$.
	\end{enumerate}
\end{theorem}
\begin{proof}
	Points \ref{item:th_ii} and \ref{item:th_iii} are equivalent by Theorem \ref{th:rank} below. The fact that \ref{item:th_i} $\Leftrightarrow$ [\ref{item:th_ii}-\ref{item:th_iii}] follows from Proposition \ref{prop:inj_sur} from Lecture~2.
\end{proof}

\begin{theorem}[Rank-nullity theorem]\label{th:rank}
	Let $L: \R^m \to \R^n$ be a linear transformation. Then
	$$
	\rank(L) + \dim(\Ker(L)) = m.
	$$
\end{theorem}

\begin{proof}
	Let us write $k = \dim(\Ker(L))$ and let us fix a basis $(x_1, \dots, x_k)$ of $\Ker(L)$. By Proposition \ref{prop:complete} one can complete this family into a basis $(x_1, \dots, x_k, x_{k+1}, \dots x_m)$ of $\R^m$.
	We will show that
	\begin{enumerate}[label=(\roman*)]
		\item \label{item:proof_i} $\Span(L(x_{k+1}), \dots, L(x_m)) = \Im(L)$.
		\item \label{item:proof_ii} the family $(L(x_{k+1}), \dots, L(x_{m}))$ is linearly independent.
	\end{enumerate}

	By proving \ref{item:proof_i} and \ref{item:proof_ii} we will get that $(L(x_{k+1}), \dots, L(x_{m}))$ is a basis of $\Im(L)$ which implies that 
	$$
	\rank(L) = \dim(\Im(L)) = m - k = m - \dim(\Ker(L)),
	$$
	hence the result.
\\

We start by proving \ref{item:proof_i}. Since $L(x_{k+1}), \dots, L(x_m)$ are all in $\Im(L)$ (which is a linear subspace) any linear combination of these vectors belongs to $\Im(L)$, hence $\Span(L(x_{k+1}), \dots, L(x_m)) \subset \Im(L)$.

Let us prove the converse inclusion.
Let $y \in \Im(L)$, which means that we can find $z \in \R^m$ such that $y = L(z)$. Let $(\alpha_1, \dots, \alpha_m) \in \R^m$ be the coordinates of $z$ in the basis $(x_1, \dots, x_m)$: $z = \alpha_1 x_1 + \cdots + \alpha_m x_m$. We have then by linearity of $L$
\begin{align*}
y = L(z) = L(\alpha_1 x_1 + \cdots + \alpha_m x_m)
= \alpha_1 L(x_1) + \cdots + \alpha_m L(x_m).
\end{align*}
Recall now that $x_1, \dots, x_k$ belong to $\Ker(L)$. Therefore $L(x_1) = \cdots = L(x_k) = 0$. We get
$$
y = \alpha_{k+1} L(x_{k+1}) + \cdots + \alpha_{m} L(x_{m}),
$$
hence $y \in \Span(L(x_{k+1}), \dots, L(x_m))$: $\Im(L) \subset \Span(L(x_{k+1}), \dots, L(x_m))$. We conclude that $\Im(L) = \Span(L(x_{k+1}), \dots, L(x_m))$.
\\

Let us now prove \ref{item:proof_ii}. To prove that $(L(x_{k+1}), \dots, L(x_m))$ are linearly independent, we consider scalars $\alpha_{k+1}, \dots, \alpha_m \in \R$ such that $\alpha_{k+1} L(x_{k+1}) + \dots + \alpha_m L(x_m) =0$. Our goal is to show that $\alpha_{k+1} = \cdots = \alpha_m = 0$. We have by linearity of $L$:
$$
0 =\alpha_{k+1} L(x_{k+1}) + \dots + \alpha_m L(x_m) = L(\alpha_{k+1} x_{k+1} + \cdots + \alpha_m x_m)
$$
which gives that $\alpha_{k+1} x_{k+1} + \cdots + \alpha_m x_m \in \Ker(L)$. Recall that $(x_1, \dots, x_k)$ is a basis of $\Ker(L)$, so there exists scalars $\alpha_1, \dots, \alpha_k$ such that $\alpha_1 x_1 + \cdots + \alpha_k x_k = \alpha_{k+1} x_{k+1} + \cdots + \alpha_m x_m$. We obtain
$$
\alpha_1 x_1 + \cdots + \alpha_k x_k - \alpha_{k+1} x_{k+1} - \cdots - \alpha_m x_m = 0
$$
which implies that $\alpha_1 = \cdots = \alpha_m = 0$ because $(x_1, \dots, x_m)$ is a basis of $\R^m$. This proves \ref{item:proof_ii}.
\end{proof}

\section{Transpose of a matrix, symmetric matrices}

\begin{definition}[Transpose]
	Let $M \in \R^{n \times m}$. We define its \emph{transpose} $M^{\sT} \in \R^{m \times n}$ by
	$$
	(M^{\sT})_{i,j} = M_{i,j}
	$$
	for all $i \in \{1, \dots, m\}$ and $j \in \{1, \dots, n\}$.
\end{definition}

\begin{remark}
	\leavevmode
	\begin{itemize}
		\item We have $(M^{\sT})^{\sT} = M$. 
		\item The mapping $M \mapsto M^{\sT}$ is linear.
	\end{itemize}
\end{remark}

We remark also that the rows of $M$ become the columns of $M^{\sT}$ and that the columns of $M$ become the rows of $M^{\sT}$. By Definition \ref{def:rank}, this gives:
\begin{proposition}
	$\rank(M) = \rank(M^{\sT})$.
\end{proposition}

\begin{proposition}\label{prop:transpose_product}
	Let $A \in \R^{n \times m}$ and $B \in \R^{m \times k}$. Then
	$$
	(AB)^{\sT} = B^{\sT} A^{\sT}.
	$$
\end{proposition}

\begin{corollary}
	If $M \in \R^{n \times n}$ is invertible, then so is $M^{\sT}$ and
	$$
	(M^{\sT})^{-1} = (M^{-1})^{\sT}.
	$$
\end{corollary}
\begin{proof}
	We compute, using Proposition \ref{prop:transpose_product}:
	$$
	M^{\sT} (M^{-1})^{\sT} = (M^{-1} M)^{\sT} = \Id_n^{\sT} = \Id_n.
	$$
	This proves that $M^{\sT}$ is invertible with inverse $(M^{-1})^{\sT}$.
\end{proof}


\begin{definition}[Symmetric matrix]
	A square matrix $A \in \R^{n \times n}$ is said to be \emph{symmetric} if
	$$
	\forall i,j \in \{1, \dots, n\}, \ A_{i,j} = A_{j,i}
	$$
	or, equivalently if $A = A^{\sT}$.
\end{definition}

The following example is fundamental:
\begin{example}[Gram matrices]
	Let $M \in \R^{k \times n}$. Then the $n\times n$ ``Gram matrix'' $A \defeq M^{\sT} M$ is symmetric.
\end{example}

	\vspace{1cm}
	\centerline{\pgfornament[width=7cm]{71}}

%\bibliographystyle{plain}
%\bibliography{./references.bib}
\end{document}
