\documentclass[11pt,nocut]{article}

\usepackage{../latex_style/packages}
\usepackage{../latex_style/notations}

\title{\vspace{-2.0cm}%
	Optimization and Computational Linear Algebra for Data Science\\
	Lecture 3: Rank}
	\author{Léo \textsc{Miolane} \ $\cdot$ \ \texttt{leo.miolane@gmail.com}}
\date{\today}

\begin{document}
\maketitle
\textbf{Warning:}
\emph{This material is not meant to be lecture notes. It only gathers the main concepts and results from the lecture, without any additional explanation, motivation, examples, figures...
}

\section{Rank of a matrix}

\begin{definition}
	We define the rank of a family $x_1, \dots, x_k$ of vectors of $\R^n$ as the dimension of its span:
	$$
	\rank(x_1, \dots, x_k) \defeq \dim (\Span(x_1, \dots, x_k)).
	$$
\end{definition}

If the vectors $x_1, \dots x_k$ are linearly independent then $\rank(x_1, \dots x_k) = k$. Indeed, in that case $(x_1, \dots, x_k)$ forms a base of $\Span(x_1, \dots, x_k)$ so $\dim(\Span(x_1, \dots, x_k)) = k$.

\begin{proposition}
	Let $x_1, \dots, x_k \in \R^n$ and write $r = \rank(x_1, \dots, x_r)$. Then there exists $i_1, \dots i_r \in \{1 \dots k\}$ such that $(x_{i_1}, \dots, x_{i_r})$ forms a basis of $\Span(x_1, \dots, x_k)$.
\end{proposition}

\begin{definition}[Rank]
	The rank of a matrix $M \in \R^{n \times m}$ is defined as the dimension of the image of $M$:
	$$
	\rank(M) = \dim(\Im(M)).
	$$
\end{definition}

Let $c_1, \dots, c_m$ be the columns of $M$, then 

\begin{proposition}
	Let $L: \R^m \to \R^n$ and $M: \R^n \to \R^k$, two linear applications. Then the following holds
	\begin{enumerate}[label=(\roman*)]
		\item $\rank(L) \leq \min(n,m)$.
		\item $\rank(ML) \leq \min(\rank(L),\rank(M))$.
	\end{enumerate}
\end{proposition}

\begin{theorem}[Rank-nullity theorem]
	Let $L: \R^m \to \R^n$ be a linear transformation. Then
	$$
	\rank(L) + \dim(\Im(L)) = m.
	$$
\end{theorem}



	\vspace{1cm}
	\centerline{\pgfornament[width=7cm]{71}}

%\bibliographystyle{plain}
%\bibliography{./references.bib}
\end{document}
