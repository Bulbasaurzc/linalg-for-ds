\documentclass{beamer}

\usepackage{../../latex_style/beamerthemeExecushares}
\usepackage{../../latex_style/notations}

\title{Lecture 1.1: Introduction}
\subtitle{Optimization and Computational Linear Algebra for Data Science}
\author{Léo Miolane}
\date{}

\setcounter{showSlideNumbers}{1}

\begin{document}
\setcounter{showProgressBar}{0}
\setcounter{showSlideNumbers}{0}

\frame{\titlepage}

\begin{frame}
	\frametitle{Contents}
	\begin{enumerate}
		\item Linear algebra \\ \textcolor{ExecusharesGrey}{\footnotesize\hspace{1em} About 2/3 of the lectures}
		\item Convex optimization  \\ \textcolor{ExecusharesGrey}{\footnotesize\hspace{1em} About 1/3 of the lectures}
		\item Overview of the lectures \\ \textcolor{ExecusharesGrey}{\footnotesize\hspace{1em} A quick look at the menu}
	\end{enumerate}
\end{frame}


\setcounter{framenumber}{0}
%\setcounter{showProgressBar}{1}
\setcounter{showSlideNumbers}{1}
\section{Linear algebra}
\begin{frame}[t]{Why linear algebra?}
	\vspace{-0.5cm}
	\begin{center}
		<< Linear algebra $\simeq$ geometry in arbitrary dimension >>
	\end{center}


	\begin{block}{Why do we need to do geometry ?}
		\begin{itemize}
			\item In many case, our data is a collection of << data points >> that are points $(x_1, \dots, x_n)$
			\item To understand the structure of our data, we have to investigate the geometry of our data points: are they divided into clusters? are they <<aligned>> ?
			\item When $n =1,2,3$, one can easily plot our data, but what about $n=10000$ ?
		\end{itemize}
	\end{block}
\end{frame}

\begin{frame}{Applications}
	You will learn linear algebra, while studying applications for data science such as:
	\vspace{0.5cm}
	\begin{itemize}
		\item Data compression \below{You will compress images using wavelets}
		\item Principal component analysis \below{Find directions along which the variance of the data is maximal}
		\item Dimensionality reduction \below{Reduce the dimension of a dataset while preserving its structure}
		\item Linear regression 
		\item Google's Page Rank and Markov chains \below{Ranking any objects that can be compared!}
		\item Clustering on networks
		\item Matrix completion
	\end{itemize}
\end{frame}


\section{Optimization}
\begin{frame}[t]{Optimization}
	In machine learning, we often have to minimize functions
	$$
	f(\theta) = {\rm Loss}\big( {\rm data}, {\rm model}_{\theta}\big)
	\quad \text{with respect to} \quad \theta \in \R^n.
	$$
	\begin{itemize}
		\item For $n=1,2$, one could plot $f$ to find the minimizer.
		\item This is intractable for larger dimension.
	\end{itemize}

	\vspace{0.2cm}
	\begin{block}{We will}
		\begin{itemize}
			\item focus on convex cost functions $f$.
			\item study gradient descent algorithms to minimize $f$.
		\end{itemize}
	\end{block}
\end{frame}

\section{Overview of the lectures}
\begin{frame}{Outline}
	\begin{enumerate}
		\item Vectors and vector spaces
		\item Linear transformations and matrices
		\item The rank
		\item Norm and inner product
		\item Eigenvalues, eigenvectors and Markov chains
		\item The spectral theorem and PCA
		\item Graphs and Linear Algebra
		\item Convex functions
		\item Linear regression
		\item Optimality conditions
		\item Gradient descent
	\end{enumerate}
\end{frame}

\begin{frame}{Further informations}


	Course's webpage:
	\vspace{0.3cm}
	\begin{center}
		\href{https://leomiolane.github.io/linalg-for-ds.html}{\large \texttt{leomiolane.github.io/linalg-for-ds.html}}
	\end{center}

\end{frame}

\end{document}
