\documentclass[11pt,nocut]{article}

\usepackage{../latex_style/packages}
\usepackage{../latex_style/notations}
\externaldocument{../lecture_02/lecture_02}
\externaldocument{../lecture_04/lecture_04}


\title{\vspace{-2.0cm}%
	Optimization and Computational Linear Algebra for Data Science\\
Lecture 8: Graphs and Linear Algebra}
\author{Léo \textsc{Miolane} \ $\cdot$ \ \texttt{leo.miolane@gmail.com}}
\date{\today}

\begin{document}
\maketitle
\textbf{Warning:}
\emph{This material is not meant to be lecture notes. It only gathers the main concepts and results from the lecture, without any additional explanation, motivation, examples, figures...
}


\section{Graphs}

We start by a formal definition of a (simple non-oriented) graph:
\begin{definition}[Graph]
	A graph $G$ is defined as a pair $V_G,E_G$ where $V=V_G$ is the set of vertices of $G$ and $E = E_G$ is the set of edges of $G$ which is a subset of $V \times V$.
	Two vertices $i,j$ are connected by an edge if $\{i,j\} \in E$. In such case we write $i \sim j$ and say that $i$ and $j$ are neighboors.
\end{definition}

\begin{definition}
	The degree of a node $i \in V$ is the number of its neighboors.
\end{definition}

In this lecture we will only consider finite graphs, where $V$ is finite. We let $n = \# V$. One can assume (up to renaming the vertices) that $V = \{1, \dots, n\}$.

\begin{definition}
	We define the adjacency matrix $A \in \R^{n \times n}$ of the graph $G$ by
	$$
	A_{i,j} = 
	\begin{cases}
		1 & \text{if} \ i \sim j \\
		0 & \text{otherwise.}
	\end{cases}
	$$
	The degree matrix of $G$ is defined by $D = \Diag(\deg(1), \dots, \deg(n))$. 
\end{definition}

Notice that $A$ is a symmetric matrix.

\section{Graph Laplacian}

\begin{definition}[Graph Laplacian]
	The Laplacian matrix of $G$ is defined as
	$$
	L = D - A.
	$$
\end{definition}

\begin{proposition}
	The matrix $L$ satisfies the following properties:
	\begin{enumerate}
		\item $L$ is symmetric and positive semi-definite.
		\item The smallest eigenvalue of $L$ is $0$, the corresponding eigenvector is the constant one vector $(1,1, \dots, 1)$.
		\item $L$ has $n$ non-negative eigenvalues $0 = \lambda_1 \leq \lambda_2 \leq \dots \leq \lambda_n$.
	\end{enumerate}
\end{proposition}

\begin{proposition}
	The graph $G$ is connected if and only if $\lambda_2 > 0$.
\end{proposition}
More generally, one can show that the multiplicity of the eigenvalue $0$ of $L$ (i.e.\ the number of $i$ such that $\lambda_i = 0$) is equal to the number of connected components of $L$.

\section{Spectral clustering with the graph Laplacian}

\begin{proposition}
	Assume that $G$ is connected. Let $v_2$ be an eigenvector associated to $\lambda_2$, the second smallest eigenvalue of $L$. Let
	$$
	W = \{ i \, | \ v_2(i) \geq 0 \}.
	$$
	Then the subgraph induced by $W$ is connected.
\end{proposition}

This indicates that 

$$
{\rm cut}(G,v) = \sum_{i \sim j} \1(v_i \neq v_i) = \frac{1}{2} \sum_{i,j} A_{i,j} (1 - v_i v_j)
$$
\vspace{1cm}
\centerline{\pgfornament[width=7cm]{71}}

%\bibliographystyle{plain}
%\bibliography{./references.bib}
\end{document}
