\documentclass{beamer}

\usepackage{../../latex_style/beamerthemeExecushares}
\usepackage{../../latex_style/notations}

\title{Lecture 2.1: Linear transformations}
\subtitle{Optimization and Computational Linear Algebra for Data Science}
\author{Léo Miolane}
\date{}

\setcounter{showSlideNumbers}{1}

\begin{document}
\setcounter{showProgressBar}{0}
\setcounter{showSlideNumbers}{0}

\frame{\titlepage}

\begin{frame}
	\frametitle{Contents}
	\begin{enumerate}
		\item Definition of a linear transformation
			\vspace{1cm}
		\item Properties of linear transformations
	\end{enumerate}
\end{frame}


\setcounter{framenumber}{0}
\setcounter{showSlideNumbers}{1}
\section{Definition}
\begin{frame}[t]{Examples}
	You already know some linear transformations from high-school !

	\vspace{0.6cm}
	\begin{columns}
		%\hspace{-2cm}
		\begin{column}{0.47\textwidth}
			\vspace{-0.25cm}
			\begin{center}
				\textbf{Symmetry}
			\end{center}
			\vspace{5cm}
		\end{column}
		\vrule
		\vrule
		\begin{column}{0.53\textwidth}
			\vspace{-0.5cm}
			\begin{center}
				\textbf{Rotation}
				\vspace{5cm}
			\end{center}
		\end{column}
	\end{columns}


\end{frame}

\begin{frame}[t]{Definition}
	Symmetries (about a line passing through the origin) and rotations (about the origin) are mappings
	$$
	\begin{array}{cccc}
		L: & \R^2 & \to & \R^2 \\
		   & v & \mapsto & L(v),
	\end{array}
	$$
	that are ``linear'':
	\vspace{0.3cm}
	\begin{block}{\bf Definition}
		A function $L: \R^m \to \R^n$ is linear if
		\begin{enumerate}
			\item for all $v,w \in \R^m$ we have $L(v + w) = L(v) + L(w)$ and
			\item for all $v \in \R^m$ and all $\alpha \in \R$ we have $L(\alpha v) = \alpha L(v)$.
		\end{enumerate}
	\end{block}
\end{frame}


\begin{frame}[t]{An example}
	\begin{itemize}
		\item $\displaystyle
			\begin{array}{cccc}
				L: & \R^2 & \to & \R^3 \\
				   & (v_1,v_2) & \mapsto & (5 v_1, \ 0 , \ v_1 + v_2)
			\end{array}
			$
			\ \ is linear
			\vspace{2cm}
		%\item
			%$\displaystyle
			%\begin{array}{cccc}
				%M: & \R^3 & \to & \R^2 \\
				   %& (x_1,x_2,x_3) & \mapsto & (x_1+x_3,2x_2)
			%\end{array}
			%$
			%\ is linear
	\end{itemize}
\end{frame}
\begin{frame}[t]{An example of a non-linear map}
		The function \quad $\displaystyle
			\begin{array}{cccc}
				F: & \R & \to & \R \\
				   & x & \mapsto & x^2
			\end{array}
			$
			\quad is \textbf{not} linear.
\end{frame}

\section{Properties}

\begin{frame}[t]{Composition of linear maps}
		\vspace{0.5cm}
	\begin{block}{Proposition}
		If $L: \R^m \to \R^n$ and $M: \R^n \to \R^k$ are both linear, then the composite function
			\vspace{-0.2cm}
		$$
			\begin{array}{cccc}
				M \circ L: & \R^m & \to & \R^k \\
						   & v & \mapsto & M(L(v))
			\end{array}
			\vspace{-0.4cm}
			$$
			is also linear.
	\end{block}
	\begin{proof}
		\vfill
		\vspace{2.5cm}
	\end{proof}
\end{frame}
\begin{frame}[t]{Basic properties}
	\begin{block}{Proposition}
		If $L: \R^m \to \R^n$ is linear, then
		\begin{itemize}
			\item $L(0) = 0$.
			\item $\displaystyle L\Big(\sum_{i=1}^k \alpha_i v_i \Big) = \sum_{i=1}^k \alpha_i L(v_i)$, for all $\alpha_i \in \R$, $v_i \in \R^m$.
		\end{itemize}
	\end{block}
	\begin{proof}
		\vfill
		\vspace{2cm}
	\end{proof}
\end{frame}

\end{document}
