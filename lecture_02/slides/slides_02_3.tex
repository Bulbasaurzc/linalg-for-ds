\documentclass{beamer}

\usepackage{../../latex_style/beamerthemeExecushares}
\usepackage{../../latex_style/notations}

\title{Lecture 2.3: Matrix product}
\subtitle{Optimization and Computational Linear Algebra for Data Science}
\author{Léo Miolane}
\date{}

\setcounter{showSlideNumbers}{1}

\begin{document}
\setcounter{showProgressBar}{0}
\setcounter{showSlideNumbers}{0}

\frame{\titlepage}


\setcounter{framenumber}{0}
\setcounter{showSlideNumbers}{1}

\begin{frame}[t]{Matrix-vector product}
	Consider a linear map $L:\R^m \to \R^n$ and its associated matrix $\widetilde{L} \in \R^{n \times m}$.
	\\
	\vspace{0.3cm}
	\textbf{Question:} Can we use the matrix $\widetilde{L}$ to compute the image $L(x)$ of a vector $x\in\R^m$ ?
	\vspace{0.3cm}

	\begin{block}{Proposition}
		For all $x\in\R^m$ coordinates of $L(x) \in \R^n$ are given by the formula:
		$$
		\big(L(x)\big)_i = \sum_{j=1}^m \widetilde{L}_{i,j} \, x_j \qquad \text{for all} \ \ i \in \{1, \dots, n\}.
		$$
	\end{block}
\end{frame}
\begin{frame}[t]{Visualizing the formula}
	\vspace{-0.9cm}
	\begin{exampleblock}{}
		\vspace{-0.4cm}
		$$
		\big(L(x)\big)_i = \sum_{j=1}^m \widetilde{L}_{i,j} \, x_j
		= \widetilde{L}_{i,1} \, x_1 + \widetilde{L}_{i,2} \, x_2 + \cdots + \widetilde{L}_{i,m} \, x_m
		$$
		\vspace{-0.4cm}
	\end{exampleblock}
\end{frame}
\begin{frame}[t]{Justification of the formula}
	\vspace{-0.9cm}
	\begin{exampleblock}{}
		\vspace{-0.4cm}
		$$
		\big(L(x)\big)_i = \sum_{j=1}^m \widetilde{L}_{i,j} \, x_j
		= \widetilde{L}_{i,1} \, x_1 + \widetilde{L}_{i,2} \, x_2 + \cdots + \widetilde{L}_{i,m} \, x_m
		$$
		\vspace{-0.4cm}
	\end{exampleblock}
\end{frame}

\begin{frame}[t]{Matrix-vector product}
	\begin{itemize}
		\item We have seen: \quad linear map $\to$ matrix
		\item We will see now: \quad matrix $\to$ linear map
	\end{itemize}

	\vspace{0.1cm}
	\begin{block}{\bf Definition}
		The linear map associated to a matrix $\widetilde{L} \in \R^{n \times m}$ is the map
		$$
		\begin{array}{cccc}
			L:& \R^m & \to & \R^n \\
			  & x & \mapsto & \widetilde{L}x
		\end{array}
		$$
		where the ``matrix-vector'' product $\widetilde{L}x \in \R^n$ is defined by
		$$
		\big(\widetilde{L}x\big)_i = \sum_{j=1}^m \widetilde{L}_{i,j} \, x_j \qquad \text{for all} \ i \in \{1, \dots, n\}.
		$$
		\vspace{-0.5cm}
	\end{block}
	%\vspace{0.2cm}
	%\textbf{Remark:} for all $x\in\R^m$, \ $L(x) = \widetilde{L}x$.
\end{frame}

\begin{frame}[t]{Matrix product}
	Let $L \in \R^{n \times m}$ and $M \in \R^{m \times k}$. 
	\vspace{1.5cm}
	\begin{block}{Definition - Proposition}
		\begin{itemize}
			\item The matrix product $LM$ is the $n \times k$ matrix of the linear map $L \circ M$.
				\vspace{0.1cm}
			\item Its coefficients are given by the formula:
		\vspace{-0.2cm}
				$$
				(LM)_{i,j} = \sum_{\ell=1}^m L_{i,\ell} M_{\ell,j} \quad \text{for all} \quad 1 \leq i \leq n, \quad 1 \leq j \leq k.
				$$
		\end{itemize}
		\vspace{-0.4cm}
	\end{block}
\end{frame}
\begin{frame}[t]{Visualizing the formula}
	\vspace{-0.9cm}
	\begin{exampleblock}{}
		\vspace{-0.4cm}
		$$
		(LM)_{i,j} = \sum_{\ell=1}^m L_{i,\ell} \, M_{\ell,j} 
		= L_{i,1} \, M_{1,j} + \cdots + L_{i,m} \, M_{m,j} 
		$$
		\vspace{-0.3cm}
	\end{exampleblock}
\end{frame}

\end{document}
