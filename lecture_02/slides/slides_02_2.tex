\documentclass{beamer}

\usepackage{../../latex_style/beamerthemeExecushares}
\usepackage{../../latex_style/notations}

\title{Lecture 2.2: Matrices}
\subtitle{Optimization and Computational Linear Algebra for Data Science}
\author{Léo Miolane}
\date{}

\setcounter{showSlideNumbers}{1}

\begin{document}
\setcounter{showProgressBar}{0}
\setcounter{showSlideNumbers}{0}

\frame{\titlepage}

\begin{frame}
	\frametitle{Contents}
	\begin{enumerate}
		\item Matrix associated to a linear transformation
		\item Matrix product
	\end{enumerate}
\end{frame}


\setcounter{framenumber}{0}
\setcounter{showSlideNumbers}{1}
\section{Matrix associated to a linear map}
\begin{frame}[t]{The key observation}

	\begin{itemize}
		\item Let $L: \R^m \to \R^n$ be a linear transformation.
		\item Let $(e_1, \dots, e_m)$ be the canonical basis of $\R^m$.
	\end{itemize}
	\vspace{0.2cm}

Then, for all $x = (x_1, \dots, x_m) \in \R^m$:
$$
L(x) = 
L\Big( \sum_{i=1}^m x_i e_i \Big) = \sum_{i=1}^m x_i L(e_i).
$$
\pause
\textbf{Conclusion}: if you give me the vectors $L(e_1), \dots, L(e_m) \in \R^n$ then, I can compute $L(x)$ for any $x \in \R^m$.

\vspace{1cm}
\begin{center}
<< One needs $n \times m$ numbers to store a the linear map $L$ on a computer >>
\end{center}

\end{frame}

\begin{frame}[t]{Matrices}
	\begin{block}{\bf Definition}
	A $n \times m$ matrix is an array (of real numbers) with $n$ rows and $m$ columns.
	We denote by $\R^{n \times m}$ the set of all $n \times m$ matrices.
\end{block}

\end{frame}
\begin{frame}{Canonical matrix of a linear map}
We can encode a linear map $L: \R^m \to \R^n$ by a $n \times m$ matrix.
\begin{block}{\bf Definition}
	The canonical matrix of $L$ is the $n \times m$ matrix (that we will write also $L$) whose columns are $L(e_1), \dots, L(e_m)$:
$$
L =
\begin{pmatrix}
	| & | & & | \\
	L(e_1) & L(e_2) & \cdots& L(e_m) \\
	| & | & & |
\end{pmatrix}
= 
\begin{pmatrix}
	L_{1,1} & L_{1,2} & \cdots & L_{1,m} \\
	L_{2,1} & L_{2,2} & \cdots & L_{2,m} \\
	\vdots & \vdots & \ddots & \vdots \\
	L_{n,1} & L_{n,2} & \cdots & L_{n,m} \\
\end{pmatrix}
$$
where we write $L(e_j) = 
\begin{pmatrix}
	L_{1,j} \\
	L_{2,j}\\
	\vdots \\
	L_{n,j}
\end{pmatrix}$.
\end{block}

\end{frame}


\begin{frame}[t]{Example \#1: identity matrix}
		The Identity map \quad $\displaystyle
			\begin{array}{cccc}
				\Id: & \R^n & \to & \R^n \\
				   & x & \mapsto & x
			\end{array}
			$
			\quad is linear.
			\\
			\vspace{0.3cm}
			\textbf{Exercise}: what is the canonical matrix of $\Id$ ?
\end{frame}
\begin{frame}[t]{Example \#2: Homothety}
	Let $\lambda \in \R$.
		The homothety map of ratio $\lambda$: 
		 $$
			\begin{array}{cccc}
				H_{\lambda}: & \R^n & \to & \R^n \\
				   & x & \mapsto & \lambda x
			\end{array}
			$$
			is linear.
			\\
			\vspace{0.3cm}
			\textbf{Exercise}: what is the canonical matrix of $H_{\lambda}$ ?
\end{frame}
\begin{frame}[t]{Example \#3: rotations in $\R^2$}
	Let $\theta \in \R$.
	The rotation $R_{\theta}: \R^2 \to \R^2$ of angle $\theta$ about the origin is linear.
			\\
			\vspace{0.3cm}
			\textbf{Exercise}: what is the canonical matrix of $R_{\theta}$ ?
\end{frame}

\section{Matrix product}

\begin{frame}[t]{Matrix-vector product}
	\begin{itemize}
		\item We have seen: \quad linear map $\to$ matrix
		\item We will see now: \quad matrix $\to$ linear map
	\end{itemize}

\begin{block}{\bf Definition}
	The linear transformation associated to a matrix $L \in \R^{n \times m}$ is the map
	$$
	\begin{array}{cccc}
		L:& \R^m & \to & \R^n \\
		  & x & \mapsto & Lx
	\end{array}
	$$
	where the ``matrix-vector'' product $Lx \in \R^n$ is defined by
	$$
	(Lx)_i = \sum_{j=1}^m L_{i,j} x_j \qquad \text{for all} \ i \in \{1, \dots, n\}.
	$$
\end{block}
\end{frame}
\begin{frame}[t]{Visualizing the formula}
	\vspace{-0.8cm}
	\begin{exampleblock}{}
	$$
	(Lx)_i = \sum_{j=1}^m L_{i,j} \, x_j
	$$
\end{exampleblock}
\end{frame}

\begin{frame}[t]{Matrix product}
	Let $L \in \R^{n \times m}$ and $M \in \R^{m \times k}$. 
	\vspace{2cm}
\begin{definition}[Matrix product]
	The matrix product $LM$ is the $n \times k$ matrix of the linear map $L \circ M$.
	His coefficients are given by the formula:
	$$
	(LM)_{i,j} = \sum_{\ell=1}^m L_{i,\ell} M_{\ell,j} \quad \text{for all} \quad 1 \leq i \leq n, \quad 1 \leq j \leq k.
	$$
\end{definition}
\end{frame}
\begin{frame}[t]{Visualizing the formula}
	\vspace{-0.8cm}
	\begin{exampleblock}{}
	\vspace{-0.3cm}
	$$
	(LM)_{i,j} = \sum_{\ell=1}^m L_{i,\ell} \, M_{\ell,j} 
	$$
	\vspace{-0.1cm}
\end{exampleblock}
\end{frame}

\end{document}
