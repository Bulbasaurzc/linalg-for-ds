\documentclass{beamer}

\usepackage{../../latex_style/beamerthemeExecushares}
\usepackage{../../latex_style/notations}

\title{Lecture 2.2: Matrices}
\subtitle{Optimization and Computational Linear Algebra for Data Science}
\author{Léo Miolane}
\date{}

\setcounter{showSlideNumbers}{1}

\begin{document}
\setcounter{showProgressBar}{0}
\setcounter{showSlideNumbers}{0}

\frame{\titlepage}


\setcounter{framenumber}{0}
\setcounter{showSlideNumbers}{1}

\begin{frame}[t]{The key observation}

	\begin{itemize}
		\item Let $L: \R^m \to \R^n$ be a linear transformation.
		\item Let $(e_1, \dots, e_m)$ be the canonical basis of $\R^m$.
	\end{itemize}
	\vspace{0.2cm}

Then, for all $x = (x_1, \dots, x_m) \in \R^m$:
$$
L(x) = 
L\Big( \sum_{i=1}^m x_i e_i \Big) = \sum_{i=1}^m x_i L(e_i).
$$
\pause
\\

\textbf{Conclusion}: if you give me the vectors $L(e_1), \dots, L(e_m) \in \R^n$ then, I am able to compute $L(x)$ for any $x \in \R^m$.

\vspace{0.7cm}
\begin{center}
<< One needs $n \times m$ numbers to store\\ the linear map $L$ on a computer >>
\end{center}

\end{frame}

\begin{frame}[t]{Matrices}
	\begin{block}{\bf Definition}
	A $n \times m$ matrix is an array (of real numbers) with $n$ rows and $m$ columns.
	We denote by $\R^{n \times m}$ the set of all $n \times m$ matrices.
\end{block}

\end{frame}
\begin{frame}{Canonical matrix of a linear map}
We can encode a linear map $L: \R^m \to \R^n$ by a $n \times m$ matrix.
\begin{block}{\bf Definition}
	The canonical matrix of $L$ is the $n \times m$ matrix (that we will write also $L$) whose columns are $L(e_1), \dots, L(e_m)$:
$$
L =
\begin{pmatrix}
	| & | & & | \\
	L(e_1) & L(e_2) & \cdots& L(e_m) \\
	| & | & & |
\end{pmatrix}
= 
\begin{pmatrix}
	L_{1,1} & L_{1,2} & \cdots & L_{1,m} \\
	L_{2,1} & L_{2,2} & \cdots & L_{2,m} \\
	\vdots & \vdots & \ddots & \vdots \\
	L_{n,1} & L_{n,2} & \cdots & L_{n,m} \\
\end{pmatrix}
$$
where we write $L(e_j) = 
\begin{pmatrix}
	L_{1,j} \\
	L_{2,j}\\
	\vdots \\
	L_{n,j}
\end{pmatrix}$.
\end{block}

\end{frame}


\begin{frame}[t]{Example \#1: identity matrix}
		The Identity map \quad $\displaystyle
			\begin{array}{cccc}
				\Id: & \R^n & \to & \R^n \\
				   & x & \mapsto & x
			\end{array}
			$
			\quad is linear.
			\\
			\vspace{0.3cm}
			\textbf{Exercise}: what is the canonical matrix of $\Id$ ?
\end{frame}
\begin{frame}[t]{Example \#2: Homothety}
	Let $\lambda \in \R$.
		The homothety map of ratio $\lambda$: 
		 $$
			\begin{array}{cccc}
				H_{\lambda}: & \R^n & \to & \R^n \\
				   & x & \mapsto & \lambda x
			\end{array}
			$$
			is linear.
			\\
			\vspace{0.3cm}
			\textbf{Exercise}: what is the canonical matrix of $H_{\lambda}$ ?
\end{frame}
\begin{frame}[t]{Example \#3: rotations in $\R^2$}
	Let $\theta \in \R$.
	The rotation $R_{\theta}: \R^2 \to \R^2$ of angle $\theta$ about the origin is linear.
			\\
			\vspace{0.3cm}
			\textbf{Exercise}: what is the canonical matrix of $R_{\theta}$ ?
\end{frame}

\end{document}
