\documentclass[11pt,nocut]{article}

\usepackage{../latex_style/packages}
\usepackage{../latex_style/notations}

\title{\vspace{-2.0cm}%
	Optimization and Computational Linear Algebra for Data Science\\
	Lecture 2: Linear transformations}
	\author{Léo \textsc{Miolane} \ $\cdot$ \ \texttt{leo.miolane@gmail.com}}
\date{\today}

\begin{document}
\maketitle
\textbf{Warning:}
\emph{This material is not meant to be lecture notes. It only gathers the main concepts and results from the lecture, without any additional explanation, motivation, examples, figures...
}

\section{Linear transformations}

%\begin{center}
%From now, we will always work with vector spaces of the form $\R^k$ for $k \in \N^*$.
%\end{center}
%\vspace{1.5mm}

\begin{definition}[Linear transformation]
	A function $L: \R^m \to \R^n$ is linear if
	\begin{enumerate}[label=(\roman*)]
		\item for all $v \in \R^m$ and all $\alpha \in \R$ we have $L(\alpha v) = \alpha L(v)$ and
		\item for all $v,w \in \R^m$ we have $L(v + w) = L(v) + L(w)$.
	\end{enumerate}
\end{definition}

Notice that $L: \R^m \to \R^n$ is linear if and only if $L(\alpha v + w) = \alpha L(v) + L(w)$ for all $v,w \in \R^m$ and all $\alpha \in \R$.

%\begin{proposition}
	%The set $\mathcal{L}(\R^m, \R^n)$ of all linear transformations from $\R^m$ to $\R^n$ is a vector space.
%\end{proposition}

\begin{proposition}
	If $L: \R^m \to \R^n$ and $M: \R^n \to \R^k$ are two linear transformations, then the composite function $M \circ L: \R^m \to \R^k$ is also linear.
\end{proposition}

\begin{remark} Let $L: \R^m \to \R^n$ be a linear transformation. Then
	\begin{itemize}
		\item $L(0) = L(0.0) = 0.L(0) = 0$.
		\item $L\big(\sum_{i=1}^k \alpha_i v_i \big) = \sum_{i=1}^k \alpha_i L(v_i)$.
	\end{itemize}
\end{remark}

Let $(e_1, \dots, e_m)$ be the canonical basis of $\R^m$.
As a consequence of the second point above we have that for all $x = (x_1, \dots, x_m) \in \R^m$:
$$
L(x) = 
L\Big( \sum_{i=1}^m x_i e_i \Big) = \sum_{i=1}^m x_i L(e_i).
$$
This means that a linear transformation $L: \R^m \to \R^n$ is uniquely characterized by the vectors $L(e_1), \dots, L(e_m) \in \R^n$.

%\begin{theorem}[Equality on a basis implies equality everywhere]\label{th:basis_cara}
	%Let $L$ and $M$ be two linear transformations from $\R^m$ to $\R^n$.
	%Let $(v_1, \dots, v_m)$ be a basis of $\R^m$ and suppose that for all $i \in \{1, \dots,m\}$ we have
	%$$
	%L(v_i) = M(v_i).
	%$$
	%Then $L = M$, i.e.\ $L(v) = M(v)$ for all $v \in \R^m$.
%\end{theorem}

\section{Matrix representation}


\begin{definition}
	A $n \times m$ matrix is an array (of real numbers) with $n$ rows and $m$ columns.
	We denote by $\R^{n \times m}$ the set of all $n \times m$ matrices.
\end{definition}

We can encode a linear transformation $L: \R^m \to \R^n$ by a $n \times m$ matrix.
\begin{definition}\label{def:canonical_matrix}
	Let $L: \R^m \to \R^n$ be a linear transformation.
	The canonical matrix of $L$ is the $n \times m$ matrix (that we will write also $L$) whose columns are $L(e_1), \dots, L(e_m)$:
$$
L =
\begin{pmatrix}
	| & | & & | \\
	L(e_1) & L(e_2) & \cdots& L(e_m) \\
	| & | & & |
\end{pmatrix}
= 
\begin{pmatrix}
	L_{1,1} & L_{1,2} & \cdots & L_{1,m} \\
	L_{2,1} & L_{2,2} & \cdots & L_{2,m} \\
	\vdots & \vdots & \ddots & \vdots \\
	L_{n,1} & L_{n,2} & \cdots & L_{n,m} \\
\end{pmatrix}
$$
where we write $L(e_j) = 
\begin{pmatrix}
	L_{1,j} \\
	L_{2,j}\\
	\vdots \\
	L_{n,j}
\end{pmatrix}$.
\end{definition}

\begin{example}[Homothety]
	Let $\lambda \in \R$. The mapping (called ``homothety of ratio $\lambda$'')
	$$
	\begin{array}{cccc}
		L: & \R^n & \to & \R^n \\
		   & x & \mapsto & \lambda x
	\end{array}
	$$
	is linear. The canonical matrix of $L$ is 
	$$
	\begin{pmatrix}
		\lambda & 0 & 0 & \cdots & 0 \\
		0 & \lambda & 0 & \cdots & 0 \\
		0 & 0 &\lambda &  \cdots & 0 \\
		\vdots & \vdots& \vdots & \ddots & \vdots \\
		0 & 0 & 0 & \cdots & \lambda
	\end{pmatrix}
	.
	$$
	In the case where $\lambda = 1$, $L$ is simply the identity, its matrix is called the identity matrix and denoted by
	$$\Id_n
	\ \defeq \
	\begin{pmatrix}
		1 & 0 & 0 & \cdots & 0 \\
		0 & 1 & 0 & \cdots & 0 \\
		0 & 0 &1 &  \cdots & 0 \\
		\vdots & \vdots& \vdots & \ddots & \vdots \\
		0 & 0 & 0 & \cdots & 1
	\end{pmatrix}.
	$$
\end{example}

Definition \ref{def:canonical_matrix} tells us that one can associate a matrix to a linear transformation. The next definition show that one can do the reverse operation: we can obtain a linear transformation from a matrix.

\begin{definition}
	The linear transformation associated to a matrix $L \in \R^{n \times m}$ is the map
	$$
	\begin{array}{cccc}
		L:& \R^m & \to & \R^n \\
		  & x & \mapsto & Lx
	\end{array}
	$$
	where the ``matrix-vector'' product $Lx \in \R^n$ is defined by
	$$
	(Lx)_i = \sum_{j=1}^m L_{i,j} x_j \qquad \text{for all} \ i \in \{1, \dots, n\}.
	$$
\end{definition}

\begin{definition}[Matrix product]
	Let $L \in \R^{n \times m}$ and $M \in \R^{m \times k}$. 
	We define the matrix product $LM$ as the $n \times k$ matrix of the linear transformation $L \circ M$.
	His coefficients are given by the formula:
	$$
	(LM)_{i,j} = \sum_{\ell=1}^m L_{i,\ell} M_{\ell,j} \quad \text{for all} \quad 1 \leq i \leq n, \quad 1 \leq j \leq k.
	$$
\end{definition}


\begin{proposition}
	Let $A \in \R^{p \times q}$, $B \in \R^{q \times r}$ and $C \in \R^{r \times s}$. Then
	$$
	(AB) C = A (BC).
	$$
\end{proposition}


\section{Kernel and image}

\begin{definition}[Kernel]
	The kernel $\Ker(L)$ (or nullspace) of a linear transformation $L: \R^m \to \R^n$ is defined as the set of all vectors $v \in \R^m$ such that $L(v) = 0$, i.e.
	$$
	\Ker(L) \defeq \big\{ v \in \R^m \, \big| \, L(v) = 0 \big\}.
	$$
\end{definition}

\begin{definition}[Image]
	The image $\Im(L)$ (or column space) of a linear transformation $L: \R^m \to \R^n$ is defined as the set of all vectors $u \in \R^n$ such that there exists $v \in \R^m$ such that $L(v) = u$. 
	$\Im(L)$ is also the Span of the columns of the matrix representation of $L$.
\end{definition}

\begin{proposition}\label{prop:inj_sur}
	$\Ker(L)$ and $\Im(L)$ are subspaces of respectively $\R^m$ and $\R^n$. We have
	\begin{center}
		$L$ injective \ $\Longleftrightarrow$ \ $\Ker(L) = \{0 \}$
	\end{center}
	and
	\begin{center}
		$L$ surjective \ $\Longleftrightarrow$ \ $\Im(L) = \R^n$.
	\end{center}
\end{proposition}
%\vspace{1cm}

\paragraph{Application: Solutions of a linear system.}
We are interested into solving the system of equations in $x = (x_1, \dots, x_m) \in \R^m$
\begin{equation}\label{eq:system}
	\left\{
		\begin{array}{ccccccccc}
		a_{1,1} x_1 &+& a_{1,2} x_2 &+& \dots &+& a_{1,m} x_m &=& y_1 \\
		\vdots &&&&&&&& \\
		a_{n,1} x_1 &+& a_{n,2} x_2 &+& \dots &+& a_{n,m} x_m &=& y_n
	\end{array}
	\right.
\end{equation}
where $a_{i,j} \in \R$ and $y = (y_1, \dots, y_n) \in \R^n$. If we define the matrix $A \in \R^{n \times m}$ by $A_{i,j} = a_{i,j}$ the system \eqref{eq:system} can be rewritten as
$$
A x = y.
$$
Solving \eqref{eq:system} precisely mean << finding the inverse image of $y$ by $A$ >>. From the definition of $\Im(A)$ we get that 
\textbf{the equation $Ax = y$ admits (at least) a solution $x_0$ if and only if $y \in \Im(A)$}.
\\

We suppose now to be in that case. We would now like to know if there are other solutions. Let $x$ be another solution to $Ax = y$. By subtraction we get 
$$
A(x - x_0) = y - y = 0.
$$
This means that $(x - x_0) \in \Ker(A)$: any solution of $Ax = y$ can therefore be written as $x = x_0 + v$ with $v \in \Ker(A)$. Conversely, one can verify easily that any vector of this form is a solution. We conclude that if the equation $Ax = y$ admits a solution $x_0$, then the set of \textbf{all} solutions is
$$
x_0 + \Ker(A) \defeq \big\{ x_0 + v \, \big| \, v \in \Ker(A) \}.
$$
In particular, \textbf{$x_0$ is the unique solution if and only if $\Ker(A) = \{0\}$}.



	\vspace{1cm}
	\centerline{\pgfornament[width=7cm]{71}}

%\bibliographystyle{plain}
%\bibliography{./references.bib}
\end{document}
