\documentclass{beamer}

\usepackage{../../latex_style/beamerthemeExecushares}
\usepackage{../../latex_style/notations}

\title{Session 1: Vector spaces}
\subtitle{Optimization and Computational Linear Algebra for Data Science}
\author{Léo Miolane}
\date{}

\setcounter{showSlideNumbers}{1}

\begin{document}
\setcounter{showProgressBar}{0}
\setcounter{showSlideNumbers}{0}

\frame{\titlepage}

\begin{frame}
	\frametitle{Contents}
	\begin{enumerate}
		\item Subspaces
		\item Linear dependency
		\item Properties of the dimension
		\item Coordinates
		\item Why do we care about all these things ? \below{Application to data science}
	\end{enumerate}
\end{frame}


\setcounter{framenumber}{0}
%\setcounter{showProgressBar}{1}
\setcounter{showSlideNumbers}{1}

\section{Questions ?}
\begin{frame}[t]{Questions ?}
	\pause
\end{frame}

\section{Subspaces}
\begin{frame}{What are the subspaces of $\R^2$ ?}
\end{frame}

\begin{frame}[t]{The span is always a subspace}
	\begin{block}{\bf Proposition}
		Let $x_1, \dots, x_k \in V$. Then, $\Span(x_1, \dots, x_k)$ is a subspace of $V$.
	\end{block}
\end{frame}

\section{Linear dependency}
\begin{frame}[t]{A useful lemma}
	\vspace{-0.4cm}
	\begin{block}{\bf Lemma}
		Let $v_1, \dots, v_n \in V$
		and let $x_1, \dots, x_k \in \Span(v_1, \dots, v_n)$.
		\\
		Then, if $k > n$,
		$x_1, \dots, x_k$ are linearly dependent.
	\end{block}
	\textbf{Abuse of language:} Instead of saying <<$x_1, \dots, x_k$ are linearly dependent>>, we should have said <<the family $(x_1, \dots, x_k)$ is linearly dependent>>.
\end{frame}

\section{Basis, dimension}
\begin{frame}[t]{The dimension is well defined!}
	\begin{block}{\bf Theorem}
		If $V$ admits a basis $(v_1, \dots, v_n)$, then every basis of $V$ has also $n$ vectors. We say that $V$ has dimension $n$ and write $\dim(V) = n$.
	\end{block}
	\begin{proof}
		\vspace{3cm}
		\vfill
	\end{proof}
\end{frame}
\begin{frame}[t]{Properties of the dimension}
	\vspace{-0.3cm}
	\begin{block}{\bf Proposition}
		Let $V$ be a vector space that has dimension $\dim(V) = n$. Then
		\begin{itemize}
			\item Any family of vectors of $V$ that are linearly independent contains at most $n$ vectors.
				\below{i.e.\ if $x_1, \dots, x_k \in V$ are linearly independent, then $k \leq n$.}
			\item Any family of vectors of $V$ that spans $V$ contains at least $n$ vectors.
				\below{i.e.\ if $x_1, \dots, x_k \in V$ are such that $\Span(x_1, \dots, x_k) = V$, then $k \geq n$.}
		\end{itemize}
	\end{block}
	\begin{proof}
		\vspace{2cm}
		\vfill
	\end{proof}
	\pause
\end{frame}

\begin{frame}[t]{Properties of the dimension}
	\vspace{-0.3cm}
	\begin{block}{\bf Proposition}
	Let $V$ be a vector space of dimension $n$ and let $x_1, \dots, x_n \in V$.
	\begin{enumerate}
		\item If $x_1, \dots, x_n$ are linearly independent, then $(x_1, \dots, x_n)$ is a basis of $V$.
		\item If $\Span(x_1, \dots, x_n) = V$, then $(x_1, \dots, x_n)$ is a basis of $V$.
	\end{enumerate}
	\end{block}

	\vspace{0.5cm}

	Very useful to show that a family of vector forms a basis!

	\vspace{0.5cm}

	\begin{proof}
		\vfill
	\end{proof}
\end{frame}

\begin{frame}[t]{An inequality}
	\begin{block}{\bf Proposition}
		Let	$U$ and $V$ be two subspaces of $\R^n$. Assume that $U \subset V$. Then
		$$
		\dim(U) \leq \dim(V) \leq n.
		$$
		If \textbf{moreover} $\dim(U) = \dim(V)$, then $U = V$.
	\end{block}
\end{frame}

\begin{frame}[t]{A bit of vocabulary}
	\begin{block}{\bf Definition}
		Let $S$ be a subspace of $\R^n$.
		\begin{itemize}
			\item We call $S$ a \emph{line} if $\dim(S) = 1$.
			\item We call $S$ an \emph{hyperplane} if $\dim(S) = n-1$.
		\end{itemize}
	\end{block}
\end{frame}

\section{Coordinates}
\begin{frame}[t]{Coordinates of a vector in a basis}
	\vspace{-0.4cm}
	\begin{block}{\bf Definition}
		If $(v_1, \dots, v_n)$ is a basis of $V$, then for every $x \in V$ there exists a unique vector $(\alpha_1, \dots, \alpha_n) \in \R^n$ such that
		$$
		x = \alpha_1 v_1 + \dots + \alpha_n v_n.
		$$
		We say that $(\alpha_1, \dots, \alpha_n)$ are the coordinates of $x$ in the basis $(v_1, \dots, v_n)$.
	\end{block}
\begin{proof}
	\vspace{2.5cm}
	\vfill
\end{proof}
\end{frame}

\begin{frame}[t]{Exercise}
	\vspace{-0.8cm}
	\begin{exampleblock}{}
		\begin{enumerate}
	\item Show that the vectors $v_1 = (1,1)$ and $v_2=(1,-1)$ form a basis of $\R^2$.
	\item Express the coordinates of $u=(x,y)$ in the basis $(v_1,v_2)$ in terms of $x$ and $y$.
		\end{enumerate}
	\end{exampleblock}
	\pause
\end{frame}

\section{Why do we care about this ?}
\begin{frame}[t]{Application to image compression}
	\begin{itemize}
	\item Image = grid of pixels
\end{itemize}
\end{frame}

\end{document}
